% Mario Latendresse, October 2, 2000.
% This is a draft documentation for the Java compiler.

\documentclass{book}


\newcommand{\func}[1]{\verb+#1+}
\newcommand{\var}[1]{\verb+#1+}
\newcommand{\file}[1]{\verb+#1+}
\newcommand{\classfile}[1]{\verb+#1+}
\newcommand{\jinst}[1]{\verb+#1+}	
\newcommand{\javaMethod}[1]{{\it #1}}
\newcommand{\javaFile}[1]{\verb+#1+}

\begin{document}

{\Large This compiler documentation is a very rough draft that should
not be considered final.}

\section{Introduction}

The Java compiler is written in Scheme. It can be run under MzScheme
and also under other Scheme implementations by adding the macros to
define the structures.

The compiler can generate bytecode, that is class files.

\section{Passes of the front-end}

The front-end makes three passes. This simplifies forward references.
Consult functions \func{first-pass}, \func{second-pass}, and
\func{third-pass}.

The first pass registers the types of the compilation unit. It builds
the basic content of the global table.

The second pass verifies the existence of extends and implements, that
is they are resolved. It also checks for cycle. If one is found, the
compiler stops.

The third pass goes through all method statements. It is clearly the
most complex.

\section{Table lookup of names}

This is a rather complex issue where the data structure of the table
has to be designed carefully. 

There are three different table structures: Ltable, Ttable, and Gtable
(see corresponding structures in \file{java_struct.scm}). The Ltable
is for local variables of a method, the Ttable is for a type, and
Gtable for the types of a compilation unit.

Three functions are used to search for names. The most general is
\func{lup}. It can search through the file system. 

\subsection{Introducing a package name}

\subsection{Looking up a qualified name}

Looking up names in Java is tricky. I think that even the current Sun
compiler (1.2.1) does not do a correct job. But I rely on it to
understand some strange cases.

\subsection{The function lup}

If the name is a true qualified name with several components, the
search starts with the first component. This is what \func{lup} does
in case1 and case2. But for case3 it looks up for a type with the full
qualified name.

\subsubsection{An error in Sun's compiler?}

The following class cannot be compile under Sun's compiler version
1.2.1.

class java { static int x; }

The compiler generates the following message:

  Superclass java.lang.Object of class java not found. 

Obviously the type 'java' makes it no longer possible to lookup the
package java.lang. This seems an error according to the last paragraph
of page 107 of section 6.8.1 of JLS. Our compiler do compile this
class since when it looks for 'java.lang.Object' it knows that it must
be a type due to the context. So there is no reference to `java'.

Furthermore, if the file \classfile{java.class}\ exists in the current
directory, Sun's compiler (version 1.2.1) no longer can access the
class \class{java.lang.Object}. So in that case a compilation of any
class will not work. This seems to be an error and I have tried to
avoid it in the current compiler. 

To remove this error, when the context explicitly calls for a type,
the function \func{lup}\ is called asking for a type (i.e. kind type).

\section{Detection of cycle in types}

It is necessary to detect cyclic definition of types. It is done after
pass two and before pass three. As soon as a cycle is found, the
compiler should stop to avoid any non terminating computation is pass
three.

The detection algorithm must go through all classes of the hierarchy
involved in the compilation unit. This is the reason for reading
classes and interfaces in libraries declared as extend and implements.

%----------------------------------------------------------------------
\section{Reading class files}

Class files are read by the module \file{readWriteClass.scm} and
transformed to an AST by \file{classToAst.scm}.

\subsection{Lazy resolution of type names}

References to types in the constant pool are not resolved when the
file is read. Otherwise a large number of class files might be read.
A type is resolved when needed. The function \func{res-T} does this
resolution.

For example, it occurs for extends when a field or method is searched
in \func{lup-T}. It also occurs for the parameters type of a method
and its result. This is one of the tricky complication when
manipulating the name of a type. A type has a name, or qualified name,
but internally it is represented as a structure (type).

The compiler tries to delay the reading of a type class files but it
is not done as much as it could. For example, when searching for
applicable methods, the resolution of parameter types could be delayed
further by first looking at the primitive types. This would prevent
trying to resolve some type name when primitive types are enough to
conclude that the method is not applicable. There is no explicit
presentation of these details in the JLS books.

%----------------------------------------------------------------------
\section{Definite assignment}

The compiler does not verify for definite assignment. 

\section{Static initializer and instance initializer}

Such initializers are not implemented in the present compiler.
Note that instance initializer is a feature of Java 1.1 only.

\section{Line numbers}

A class file may contain a line table attribute on a code
attribute. This line table attribute may be used by the JVM to print
the source line number of a run-time error.

In the current version of the compiler, a line table attribute is
produced. See function \func{attribute-sourceFile}. The Scheme code
generates '(lineNumber i) elements in the bytecode which is treated at
the end by \func{code->code-lineNumberTable}.

\section{Unreachable statements}

The notion of unreachable statements is presented in JLS section
14.19. We have sligthly diverted from these rules as noted below.

The function ver-stmt has one boolean parameter to help detect
unreachable statements. If true, it means that either the statement to
analyze is reachable or an error message has been given stating that
it is unreachable. It does not simply says that the statement is
reachable or unreachable since that could generates too many error
messages in the case of multiple adjacent unreachable statements.

The technique to detect unreachable statements can be summarized as
follows.

An instruction that follows a break, continue, return or throw is
unreachable. The first instruction of a block that is unreachable is
considered unreachable. The inside statement of a loop (for, do,
while) is unreachable if its conditional expression is a constant
expression evaluated to false. The instruction following a loop
statement is unreachable if its conditional value is a constant
expression evaluating to true and there is no break that goes outside
the loop. A reachable break without a label goes outside the loop, and
a reachable break with a label attached to the loop does it too. All
other breaks do not go outside the loop.

An empty statement is never considered unreachable. Actually, if it is 
unreachable according to 14.19, its unreachability is passed to the
statement following the empty statement.

Additionally, catch expressions have to be verified to detect
unreachable catch blocks. A catch block is unreachable if its exception
expression is the same type or a supertype of an earlier catch block's
exception expression or that no exception is raised in the try block
assignable to it. Note that a RuntimeException is always considered
possible without explicitly being thrown.

The if statement is not treated like a loop, that is the conditional
expression does not affect reachable statements. This allows
``conditional compilation.''

Sun's compiler behavior is such that an unreachable statement is
considered to complete normally. It implies that if a statement is
unreachable the following one will not a priori be considered
unreachable. This allows a resetting of the algorithm to further
detect unreachable statements.

After a statement is analysed three values are returned: a list of
thrown exceptions and two boolean indicators to signal normal
completion and the presence of a break statement without a label.

\section{Field access}

\subsection{Static field}

A reference to a non static field of a class, with implicit or
explicit `this', in a static context is an error.

There is also the other direction. The case A.x where A is a type only
works if x is a static field.  This is verified by the function lup.

Let suppose the access a.b, where a is of type T and b is a static
field of T. The code for a.b does not involve a, although it is used
to construct the type of the access (the type of b). Similarly, a
long sequence a.b.c.d where d is static does not require code to
access a.b.c.

An access V[e].a where a is a static field would not require an
evaluation of expression e and an access to V, but the Sun's compiler
1.2.1 does generate code to evaluate it. The value is then discarded
and a \jinst{getstatic}\ is executed. Our compiler does not evaluate
it.

\section{Evaluation of constant expressions}

Constant expressions must be recognised to properly report semantic
errors.  For example, in the switch instructionm, case values must be
constant expressions.

The evaluation of constant expressions can be tricky as it is
necessary to emulate part of the operations available on the virtual
machine. For example, the addition of integers must be performed
according to the truncation of two's complement encoding. For 
example 2*6536*65536 results in the value 0, without an error or
warning from the compiler.

A local variable or field act as a constant only if it is declared
final. By the time it is used it must have a definite value. According
to Java 1.0, this value must be provided as an initializer.

Characters act as integer values. Therefore 'a'+'a' is correct and
results as 194. As soon as an arithmetic operation is performed using
a character, that character is considered to be an int.

String concatenation, using +, should follow the rules of the Java
language. Therefore, 9+"a" gives "9a", whereas 'a'+"a" gives "aa".
Note though that ('a'+'a')+"a" results in "194a".

Casts can be applied on intermediate values. So (byte)256+(byte)256 is
0 not 512. Every cast operations should be performed to truncate
values.

There are further complications involving NaN, infinities, and minus
zero. 

Chapter 15 of JLS lay down the rules of constant expression evaluation.

See files testConstExpr.java and runConstExpr.java for a list of
test cases.

\section{Method definition}

\subsection{The maximum height of the stack}

The maximum height of the stack is not actually calculated, it is
arbitrarily set to 1000.

\subsection{Constructor declaration}

When there is no constructor of any signature in a class, one, with
signature (), must be added by the compiler. This is done in pass
two. It is preferable to add default constructors (<init>) before the
third walk, since it might refer to these constructors, for example
when a subclass constructor implicitly or explicitly refers to it.

\section{Method invocation}

\subsection{Constructor}

A constructor cannot be called as a method but only during instance
creation (new) or as a call using super. This fact allows us to change
all constructor names to <init> in the table and its type. This is
done once the second pass is finished.

\subsection{Super invocation}

A super invocation super(...) can only be at the very beginning of the
constructor, even before any declaration. Moreover, there can only be
one. This allows a simple manipulation when initializer code is
inserted after the super call but before the code of the constructor
(see \func{block->jvm}).

\subsection{Special invocation}

Private and <init> methods, as well as calls of the form super.f(...)
must be done with \jinst{invokespecial}.

\subsection{Static method}

A static method cannot access a non static field of its class.
Therefore the form this.a or simply a where a is a non static field is
an error. 

\subsection{The return statement}

All method should end by the execution of a \jinst{return} instruction.

If the return is nested in a try with a finally, it must first execute
the finally block before returning. Actually there might be several
finally blocks to execute. 

Note the following special situation: If a return statement returns a
value it must be calculated before calling the finally blocks. To
implement this semantic, the expression is first evaluated and stored
in a local variable, the finally blocks are called (\jinst{jsr}), then the
saved value is put back on stack before returning (return).

\section{Initializers of fields, block, and local variables}

See the function \verb+type->fields&Init+.

A field may have an initializer. If it is a constant expression, it is
an attribute in the constant pool; otherwise it is an expression or an
array initializer\footnote{An array of constants is not a constant
inserted in the constant pool.}.

If a field is static, its initializer code is in the method
\verb+<clinit>+, otherwise it is in every constructor between the call
of its super class constructor and the code of the constructor. A
static field is initialized with \verb+putstatic+, whereas a non
static field uses \verb+putfield+\footnote{Since the local variable 0
contains \verb+this+, a \verb+aload_0+ will be used in combination of
\verb+putfield+.}.

Since a non static field with an initializer may appear after a class
constructor, it is easier to scan all fields before the methods to
gather the initializer codes to insert in the constructors.

The method \verb+<clinit>+ is constructed from all static
initializers. Most of this code is created while scanning the fields
declaration by \verb+type->fields&Init+. It simply concatenates all
initializer code in the order they appear in the source code. It is
important to maintain that order since some code may depend on
previous values. After that a return instruction is added at the end
by \verb+type->methods+.

An interface may have only one method, namely \verb+<clinit>+,
containing all static initializers.

For a local variable, the code of its initializer is in the method
itself before the code proper. If a local variable has a constant
expression can it be safely removed from the frame?

A block of code can be a member of a class (not for an interface).  If
it is declared static, its code goes into <clinit>, otherwise it goes
in every constructor of the class. The block of codes must be gathered
in the order they occur in the source file. No return statement may
appear in such a block. This is verified by \func{ver-stmt} using the
parameter \var{init?}. In a static block, the prefix \var{this}\ and
\var{super}\ cannot be used. This is enforced by the usual
verification of non-static references in a static context.

It has not yet been implemented to verify the circularity or the
forward reference to a variable in an initializer.


\section{Synchronized statement}

The synchronized statement uses the JVM instructions
\jinst{monitorenter} and \jinst{monitorexit}. The lock of the
synchronized is applied to an object. This object is stored in a local
variable to be used by the two JVM instructions. It is the
responsibility of the generated code to release the lock in all
circumstances, including raised exception, break statements, etc.

It is necessary to generate an entry in the exception table to catch
any exception type for the range of the synchronized statement. This
small exception handler release the lock of the synchronized object
and rethrow the exception.

For branches outside the synchronized statement, done by continue,
break, and return statements, a release of the lock must be done.

\section{The try statement}

The generation of code for the try statement requires special
attention for the continue, break, and return statements -- with or
without a label -- when a finally clause is present. In that case,
these statements are preceded by a call (jsr) to the finally
block. The finally block commonly ends by a `ret' statement, therefore
back to the execution of those statements.

\subsection{catch}

The catch block has one parameter. Its name is always different than
all other local variable and parameters. At the entry of the catch the
exception object is on the top of the stack. It is immediately stored
in the parameter of the catch and the block is executed. At the end
there is a call (jsr) to the finally block, if present. Afterward a
branch is done outside the try statement.

\subsection{finally}

The finally block is always called by a jsr instruction. It starts by
storing the return adress in a local variable. It always ends by the
ret instruction specifying the local variable containing the return
address. 

It is called even if there is a break or continue going outside the
try. (This is not properly handled by the pizza compiler. See
runTry.java.)

The compiler generates a very short exception handler, inserted after
all catch blocks, to catch all possible exceptions. This block simply
calls the finally block and throw a second time the exception. A
typical block is

\begin{verbatim}
 astore_2   ; stores the exception object
 jsr 85     ; call finally
 aload_2    ; throw the exception,
 athrow     ;   a second time.
\end{verbatim}

\section{Labels}

A label may appear on any statement. They are not inserted in any
table.  Conceptually they must be in a different name space than all
other names. Label are not inserted in the label, but in a separate
list.  The list of active labels is one of the parameter of
\func{ver-stmt}.

\subsection{Break and continue}

The label of a continue must be on a loop (for, do, and while). The
label of a break can be on any statement. In both cases the label must
be on an enclosing instruction. Therefore, a label on a simple
instruction, as an assignment, is useless.

\section{Assignment expression}

An assignment of the form v op= e is similar to v = v op e, but there
is a difference. Actually it is equivalent to v = (type of v)(v op e).
That is, there is an implicit cast applied to the expression after its
evaluation. This is mostly useful in the case that a narrowing must be
done in order to have a correctly typed assignment.

For example if variable v is an int, the assignment v += 1L is legal,
eventhough 1L is a long.

%----------------------------------------------------------------------
\section{Modifiers}


\subsection{Accessibility}

The type accessibility discuss in this section applies to members
only: fields, methods, and constructors.

There are three explicit accessibility modifiers: public, protected,
and private. There is a fourth implicit accessibility: the absence of
any modifier, called the default accessibility.

A member that is private can only be referenced in the class where it
is defined. Even a `super.x' cannot refer to a private member x.

There is a difference of interpretation of accessibility between
fields, methods, and constructors.

A field declaration in a subtype can override the same field of its
supertype with a different accessibility.

\subsubsection{Field accessibility}

The simplest accessibility is public. Its verification does not
require to know the origin of the reference and source of the field.
All public fields are accessible, regardless of the hierachies and
packages.


Protected accessibility is tricky. Consider for example two classes
declared in two packages:

\begin{verbatim}
   package A;
   class A { protected int a; }
   
   package B;
   class B extends A.A { int m(A.A x){ return x.a; } };
\end{verbatim}

Although B.B is a subclass of A.A, the field x.a is not accessible in
method m. (If B were in the same package as A, field a would be
accessible.) Therefore, the class name containing the reference and
class name where the field is declared are not sufficient to detect
such a case: the origin of the reference is indeed a subclass of the
class declaring the field, but it is not accessible.

For protected, private, and default accessibility, a verification can
be done with the full class name where the reference occurs and the
full class name of the field accessed.

A field access can be ambiguous. This is detected by function lupT-f.
To do so, it is essential that the accessibility be taken into
account. 


\subsubsection{Method accessibility}

\subsubsection{Constructor accessibility}

\section{Contract Java}

The compiler has been extended to handle contracts on methods and
constructors. These contracts can be specified in interfaces and
classes.

New lexemes and rules were added to the lexer and grammar
descriptions. Consult files java.lex and java.lal under `contract'.

The structure tyM has a component `contract' to keep track of methods
with contracts and new methods added to handle them.

For every method \javaMethod{m}\ that has a contract, a new method
\javaMethod{m}\verb+<c>+ is added. This new method is called a
wrapper. It implements the verification of contracts and the
hierarchy. Any call to \javaMethod{m}\ should now call
\javaMethod{m}\verb+<c>+. Old code that were calling \javaMethod{m}\
should be recompiled to effectively call the wrapper.

The redirection of the call is done when the bytecode is generated,
not at semantic analysis.

The wrapper is built by the semantic analyses. It is mainly an AST
construction followed by a semantic analysis.  Since the wrapper makes
a call to the original method \javaMethod{m}\, the bytecode generator
must recognised that fact and not make a call to the wrapper
itself. The field contract of structure tyM is used by the bytecode
generator to make the appropriate decision.

\subsection{Hierarchy checking}

If the method with a contract overrides a method in the supertype, two
new methods are added to verify upwardly the hierarchy.

If the overriden method has no pre-condition, it is assumed to be
`true'. If it has no post-condition it is assumed to be `false'.

\end{document}