\input tex2page
\input css.t2p

\title{scmxlate}

\ignorenextinputtimestamp

\centerline{\urlh{scmxlate.tar.gz}{\htmlonly
Download \endhtmlonly Version \input version }}

\centerline{\urlh{../index.html}{Dorai Sitaram}}

\medskip

\p{scmxlate} is a configuration tool for software
packages written in Scheme.   

\p{scmxlate} provides the package writer with a
strategy for programmatically specifying the changes
required to translate the package for a variety of
Scheme dialects and Common Lisp and a variety of
operating systems.  The end-user simply loads {\em one}
file into his Scheme or Common Lisp, which triggers the
entire configuration process with little or no
further intervention. 

Thus, there are two types of user for \p{scmxlate}:
(1) The package writer who uses the \p{scmxlate}
methodology to specify the configuration details for
his package, and (2) the end-user of that package, who uses
\p{scmxlate} to actually perform the configuration for
his system.  

The package writer is still required to know a lot more
about the configuration process than the end-user, even
with \p{scmxlate} helping the former.  (The Common Lisp half
of \p{scmxlate} uses \urlh{scm2cl.html}{\p{scm2cl}},
which is included in the \p{scmxlate} distribution.)
The advantage to using \p{scmxlate} is that his several
end-users can each configure his product to their
different systems by following the same simple
step.  

For the moment, I am the only \p{scmxlate} user of type
1.  If there is interest, I will document in due course
how other package writers may use \p{scmxlate}.
The rest of this documentation is for \p{scmxlate}
users of type 2.

\subsection*{How to configure a package that's been set
up with \p{scmxlate}}

Scenario: You are an end-user who has just downloaded a
Scheme package, say,
\urlh{../tex2page/tex2page-doc.html}{\p{tex2page}}.
The package author has included the \p{scmxlate}
configuration details in the package.  What do you do?

First, you need to have \p{scmxlate} installed on {\em
your} system.  Get the \p{scmxlate} tarball and unpack
it, creating a directory called \p{scmxlate}.  Place
this directory in its entirety in a place that is
convenient to you.  Among the files in the directory
\p{scmxlate} is the file \p{scmxlate.scm}.  Note down
its {\em full} pathname so you can refer to it from
anywhere on your filesystem. 
Just to make it concrete, let's assume you put the
\p{scmxlate} directory in \p{/usr/local/lib}.  Then the
full pathname to remember is

\p{
/usr/local/lib/scmxlate/scmxlate.scm
}


Now to configure the \p{tex2page} package.  Unpack it
and \p{cd} to its directory.  

The files to be translated are listed in a file
\p{files-to-be-translated.scm} in the subdirectory
\p{dialects}.  For each such file {\em filename}, there
may (but not necessarily) be a user-configurable file
\p{scmxlate-}{\em filename} in the top directory.  If
the package suggests you edit them, do so.

Start your Scheme or
Common Lisp in that directory (being in that directory
is important!).  In your Scheme (or Common Lisp), type 


\q{
(load "/usr/local/lib/scmxlate/scmxlate.scm")
}

\n where the \q{load} argument is of course the  full
pathname of the file \p{scmxlate.scm} on your system.

\p{scmxlate} may ask you a few questions.  A
choice of answers will be provided, so you don't need
to be too creative.  When \p{scmxlate} finishes, you
will be left with a version of the package tailormade
for you.


\bye

