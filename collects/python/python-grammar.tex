% table column lengths were modified in:
%  list_for, compound_stmt, if_stmt, while_stmt, for_stmt, try_stmt,
%  and except_clause_list_plus
\documentclass[12pt]{article}
\title{The Python Programming Language}
\author{Daniel Silva}
\date{\today}
\usepackage{tex2page}
\begin{document}
\maketitle
\tableofcontents
\newpage

\section{Program}
The grammar for Python programs is presented here.  The empty string is represented by $\varepsilon$.

The starting non-terminal is usually {\bf \verb+<+file\_input~\ref{filezzzinputb}\verb+>+}, but {\bf \verb+<+eval\_input~\ref{evalzzzinputb}\verb+>+} is used instead by the \verb|eval| function.

If a non-terminal is named {\bf nonterm\_list\_plus}, it is assumed that it defines the regular expression {\bf nonterm+}.

\subsection{ident}
\label{identb}
\begin{tabular}{lcl}
{\bf \verb+<+ident\verb+>+} & ::=  & NAME \\
\end{tabular} \\

The NAME token is a valid Python identifier.  A valid Python identifier is an alphanumeric sequence of characters starting with a letter.  The underscore character (\_) counts as a letter.  The language is case-sensitive.

\subsection{file\_input}
\label{filezzzinputb}
The following non-terminal ({\bf file\_{}input}) is the result of the parser in the following situations:
\begin{itemize}
\item when parsing a complete Python program (from a file or from a string);
\item when parsing a module;
\item when parsing a string passed to the exec statement.
\end{itemize}

\begin{tabular}{lcl}
{\bf \verb+<+file\_input\verb+>+} & ::=  & $\varepsilon$ \\
 & \verb+|+  & {\bf \verb+<+file\_input~\ref{filezzzinputb}\verb+>+}  NEWLINE \\
 & \verb+|+  & {\bf \verb+<+file\_input~\ref{filezzzinputb}\verb+>+}  {\bf \verb+<+stmt~\ref{stmtb}\verb+>+}  \\
\end{tabular}

\subsection{eval\_input}
\label{evalzzzinputb}
Eval Input is used by the Python \verb+eval+ and \verb+input+ functions.  Not all statements (\ref{stmtb}) are allowed; only expression statements (\ref{exprzzzstmtb}) are valid in this case, since the others are executed for side-effects. \\

\begin{tabular}{lcl}
{\bf \verb+<+eval\_input\verb+>+} & ::=  & {\bf \verb+<+tuple\_or\_test~\ref{tuplezzzorzzztestb}\verb+>+}  \\
 & \verb+|+  & {\bf \verb+<+eval\_input~\ref{evalzzzinputb}\verb+>+}  NEWLINE \\
\end{tabular}

\section{Statements}

\subsection{stmt}
\label{stmtb}
\begin{tabular}{lcl}
{\bf \verb+<+stmt\verb+>+} & ::=  & {\bf \verb+<+simple\_stmt~\ref{simplezzzstmtb}\verb+>+}  \\
 & \verb+|+  & {\bf \verb+<+compound\_stmt~\ref{compoundzzzstmtb}\verb+>+}  \\
\end{tabular} \\

Simple statements (\ref{simplezzzstmtb}) span a single line (no new indentation levels).  Compound statements (\ref{compoundzzzstmtb}) may span multiple lines. \\

\subsection{simple\_stmt}
\label{simplezzzstmtb}
\begin{tabular}{lcl}
{\bf \verb+<+simple\_stmt\verb+>+} & ::=  & {\bf \verb+<+small\_stmt~\ref{smallzzzstmtb}\verb+>+}  NEWLINE \\
 & \verb+|+  & {\bf \verb+<+small\_stmt~\ref{smallzzzstmtb}\verb+>+}  \verb|;| NEWLINE \\
 & \verb+|+  & {\bf \verb+<+small\_stmt~\ref{smallzzzstmtb}\verb+>+}  \verb|;| {\bf \verb+<+simple\_stmt~\ref{simplezzzstmtb}\verb+>+}  \\
\end{tabular}

\subsection{small\_stmt}
\label{smallzzzstmtb}
\begin{tabular}{lcl}
{\bf \verb+<+small\_stmt\verb+>+} & ::=  & {\bf \verb+<+expr\_stmt~\ref{exprzzzstmtb}\verb+>+}  \\
 & \verb+|+  & {\bf \verb+<+print\_stmt~\ref{printzzzstmtb}\verb+>+}  \\
 & \verb+|+  & {\bf \verb+<+del\_stmt~\ref{delzzzstmtb}\verb+>+}  \\
 & \verb+|+  & {\bf \verb+<+pass\_stmt~\ref{passzzzstmtb}\verb+>+}  \\
 & \verb+|+  & {\bf \verb+<+flow\_stmt~\ref{flowzzzstmtb}\verb+>+}  \\
 & \verb+|+  & {\bf \verb+<+import\_stmt~\ref{importzzzstmtb}\verb+>+}  \\
 & \verb+|+  & {\bf \verb+<+global\_stmt~\ref{globalzzzstmtb}\verb+>+}  \\
 & \verb+|+  & {\bf \verb+<+exec\_stmt~\ref{execzzzstmtb}\verb+>+}  \\
 & \verb+|+  & {\bf \verb+<+assert\_stmt~\ref{assertzzzstmtb}\verb+>+}  \\
\end{tabular}

\subsection{expr\_stmt}
\label{exprzzzstmtb}
\begin{tabular}{lcl}
{\bf \verb+<+expr\_stmt\verb+>+} & ::=  & {\bf \verb+<+test~\ref{testb}\verb+>+}  {\bf \verb+<+augassign~\ref{augassignb}\verb+>+}  {\bf \verb+<+tuple\_or\_test~\ref{tuplezzzorzzztestb}\verb+>+}  \\
 & \verb+|+  & {\bf \verb+<+testlist\_list\_plus~\ref{testlistzzzlistzzzplusb}\verb+>+}  \\
\end{tabular} \\

An expression statement consists of either an in-place variable mutation operation; an assignment, which can define or mutate variables; or an expression (\ref{tuplezzzorzzztestb}), which will be displayed in the output of the interpreter.

\subsection{testlist\_list\_plus}
\label{testlistzzzlistzzzplusb}
\begin{tabular}{lcl}
{\bf \verb+<+testlist\_list\_plus\verb+>+} & ::=  & {\bf \verb+<+tuple\_or\_test~\ref{tuplezzzorzzztestb}\verb+>+}  \\
 & \verb+|+  & {\bf \verb+<+tuple\_or\_test~\ref{tuplezzzorzzztestb}\verb+>+}  \verb|=| {\bf \verb+<+testlist\_list\_plus~\ref{testlistzzzlistzzzplusb}\verb+>+}  \\
\end{tabular} \\

A tuple is a list of tests.  A ``test'' is an expression or a boolean expression.  i.e., a test is either an expression x, ``x or y'', ``x and y'', or ``not x''.

\subsection{augassign}
\label{augassignb}
\begin{tabular}{lcl}
{\bf \verb+<+augassign\verb+>+} & ::=  & \verb|+=| \\
 & \verb+|+  & \verb|-=| \\
 & \verb+|+  & \verb|*=| \\
 & \verb+|+  & \verb|/=| \\
 & \verb+|+  & \verb|%=| \\
 & \verb+|+  & \verb|&=| \\
 & \verb+|+  & \verb+|=+ \\
 & \verb+|+  & \verb|^=| \\
 & \verb+|+  & \verb|<<=| \\
 & \verb+|+  & \verb|>>=| \\
 & \verb+|+  & \verb|**=| \\
 & \verb+|+  & \verb|//=| \\
\end{tabular}

\subsection{print\_stmt}
\label{printzzzstmtb}
\begin{tabular}{lcl}
{\bf \verb+<+print\_stmt\verb+>+} & ::=  & print {\bf \verb+<+test\_list~\ref{testzzzlistb}\verb+>+}  \\
 & \verb+|+  & print \verb|>>| {\bf \verb+<+test\_list~\ref{testzzzlistb}\verb+>+}  \\
\end{tabular} \\

http://www.python.org/doc/current/ref/print.html

\verb|print| has an extended form, sometimes referred to as ``print chevron.'' 
In this form, the first expression after the \verb+>>+ must evaluate to a ``file-like'' 
object, specifically an object that has a write() method as described above, or \verb|None|.

\subsection{test\_list}
\label{testzzzlistb}
\begin{tabular}{lcl}
{\bf \verb+<+test\_list\verb+>+} & ::=  & $\varepsilon$ \\
 & \verb+|+  & {\bf \verb+<+test~\ref{testb}\verb+>+}  \\
 & \verb+|+  & {\bf \verb+<+test~\ref{testb}\verb+>+}  \verb|,| {\bf \verb+<+test\_list~\ref{testzzzlistb}\verb+>+}  \\
\end{tabular}

\subsection{del\_stmt}
\label{delzzzstmtb}
http://www.python.org/doc/current/ref/del.html

\begin{tabular}{lcl}
{\bf \verb+<+del\_stmt\verb+>+} & ::=  & del {\bf \verb+<+target\_tuple\_or\_expr~\ref{targetzzztuplezzzorzzzexprb}\verb+>+}  \\
\end{tabular}

\subsection{pass\_stmt}
\label{passzzzstmtb}
\begin{tabular}{lcl}
{\bf \verb+<+pass\_stmt\verb+>+} & ::=  & pass \\
\end{tabular}

\subsection{flow\_stmt}
\label{flowzzzstmtb}
A flow statement directs or modifies program flow. \\

\begin{tabular}{lcl}
{\bf \verb+<+flow\_stmt\verb+>+} & ::=  & {\bf \verb+<+break\_stmt~\ref{breakzzzstmtb}\verb+>+}  \\
 & \verb+|+  & {\bf \verb+<+continue\_stmt~\ref{continuezzzstmtb}\verb+>+}  \\
 & \verb+|+  & {\bf \verb+<+return\_stmt~\ref{returnzzzstmtb}\verb+>+}  \\
 & \verb+|+  & {\bf \verb+<+raise\_stmt~\ref{raisezzzstmtb}\verb+>+}  \\
 & \verb+|+  & {\bf \verb+<+yield\_stmt~\ref{yieldzzzstmtb}\verb+>+}  \\
\end{tabular}

\subsection{break\_stmt}
\label{breakzzzstmtb}
\begin{tabular}{lcl}
{\bf \verb+<+break\_stmt\verb+>+} & ::=  & break \\
\end{tabular} \\

http://www.python.org/doc/current/ref/break.html

\verb+break+ may only occur syntactically nested in a for or while loop, 
but not nested in a function or class definition within that loop.   

\subsection{continue\_stmt}
\label{continuezzzstmtb}
\begin{tabular}{lcl}
{\bf \verb+<+continue\_stmt\verb+>+} & ::=  & continue \\
\end{tabular} \\

http://www.python.org/doc/current/ref/continue.html

\verb|continue| may only occur syntactically nested in a for or while loop, 
but not nested in a function or class definition or try statement within 
that loop.

\subsection{return\_stmt}
\label{returnzzzstmtb}
\begin{tabular}{lcl}
{\bf \verb+<+return\_stmt\verb+>+} & ::=  & return {\bf \verb+<+tuple\_or\_test~\ref{tuplezzzorzzztestb}\verb+>+}  \\
 & \verb+|+  & return \\
\end{tabular} \\

http://www.python.org/doc/current/ref/return.html

\verb|return| may only occur syntactically nested in a function definition, 
not within a nested class definition.  In a generator function, 
the return statement is not allowed to include an expression list (\ref{tuplezzzorzzztestb}). In that context, a bare return indicates that the generator is done and will cause \verb|StopIteration| to be raised. 

\subsection{raise\_stmt}
\label{raisezzzstmtb}
\begin{tabular}{lcl}
{\bf \verb+<+raise\_stmt\verb+>+} & ::=  & raise \\
 & \verb+|+  & raise {\bf \verb+<+test~\ref{testb}\verb+>+}  \\
 & \verb+|+  & raise {\bf \verb+<+test~\ref{testb}\verb+>+}  \verb|,| {\bf \verb+<+test~\ref{testb}\verb+>+}  \\
 & \verb+|+  & raise {\bf \verb+<+test~\ref{testb}\verb+>+}  \verb|,| {\bf \verb+<+test~\ref{testb}\verb+>+}  \verb|,| {\bf \verb+<+test~\ref{testb}\verb+>+}  \\
\end{tabular} \\

http://www.python.org/doc/current/ref/raise.html

\verb|raise| may have up to three arguments, the first being the type of the exception,
the second being its value, and the third being a traceback.

\subsection{yield\_stmt (NOT YET IMPLEMENTED)}
\label{yieldzzzstmtb}
\begin{tabular}{lcl}
{\bf \verb+<+yield\_stmt\verb+>+} & ::=  & yield {\bf \verb+<+tuple\_or\_test~\ref{tuplezzzorzzztestb}\verb+>+}  \\
\end{tabular} \\

http://www.python.org/doc/current/ref/yield.html

The \verb|yield| statement is only used when defining a generator function, 
and is only used in the body of the generator function. Using a yield 
statement in a function definition is sufficient to cause that definition 
to create a generator function instead of a normal function.

\subsection{import\_stmt (NOT YET IMPLEMENTED)}
\label{importzzzstmtb}
\begin{tabular}{lcl}
{\bf \verb+<+import\_stmt\verb+>+} & ::=  & {\bf \verb+<+import\_stmt1~\ref{importzzzstmt1b}\verb+>+}  \\
 & \verb+|+  & from {\bf \verb+<+dotted\_name~\ref{dottedzzznameb}\verb+>+}  import \verb|*| \\
 & \verb+|+  & from {\bf \verb+<+dotted\_name~\ref{dottedzzznameb}\verb+>+}  import {\bf \verb+<+import\_stmt2~\ref{importzzzstmt2b}\verb+>+}  \\
\end{tabular} \\

http://www.python.org/doc/current/ref/import.html

\subsection{import\_stmt1 (NOT YET IMPLEMENTED)}
\label{importzzzstmt1b}
\begin{tabular}{lcl}
{\bf \verb+<+import\_stmt1\verb+>+} & ::=  & import {\bf \verb+<+dotted\_as\_name~\ref{dottedzzzaszzznameb}\verb+>+}  \\
 & \verb+|+  & {\bf \verb+<+import\_stmt1~\ref{importzzzstmt1b}\verb+>+}  \verb|,| {\bf \verb+<+dotted\_as\_name~\ref{dottedzzzaszzznameb}\verb+>+}  \\
\end{tabular}

\subsection{import\_stmt2 (NOT YET IMPLEMENTED)}
\label{importzzzstmt2b}
\begin{tabular}{lcl}
{\bf \verb+<+import\_stmt2\verb+>+} & ::=  & {\bf \verb+<+import\_as\_name~\ref{importzzzaszzznameb}\verb+>+}  \\
 & \verb+|+  & {\bf \verb+<+import\_as\_name~\ref{importzzzaszzznameb}\verb+>+}  \verb|,| {\bf \verb+<+import\_stmt2~\ref{importzzzstmt2b}\verb+>+}  \\
\end{tabular}

\subsection{import\_as\_name (NOT YET IMPLEMENTED)}
\label{importzzzaszzznameb}
\begin{tabular}{lcl}
{\bf \verb+<+import\_as\_name\verb+>+} & ::=  & {\bf \verb+<+ident~\ref{identb}\verb+>+}  NAME {\bf \verb+<+ident~\ref{identb}\verb+>+}  \\
 & \verb+|+  & {\bf \verb+<+ident~\ref{identb}\verb+>+}  \\
\end{tabular}

\subsection{dotted\_as\_name (NOT YET IMPLEMENTED)}
\label{dottedzzzaszzznameb}
\begin{tabular}{lcl}
{\bf \verb+<+dotted\_as\_name\verb+>+} & ::=  & {\bf \verb+<+dotted\_name~\ref{dottedzzznameb}\verb+>+}  NAME {\bf \verb+<+ident~\ref{identb}\verb+>+}  \\
 & \verb+|+  & {\bf \verb+<+dotted\_name~\ref{dottedzzznameb}\verb+>+}  \\
\end{tabular}

\subsection{dotted\_name (NOT YET IMPLEMENTED)}
\label{dottedzzznameb}
\begin{tabular}{lcl}
{\bf \verb+<+dotted\_name\verb+>+} & ::=  & {\bf \verb+<+ident~\ref{identb}\verb+>+}  \\
 & \verb+|+  & {\bf \verb+<+ident~\ref{identb}\verb+>+}  \verb|.| {\bf \verb+<+dotted\_name~\ref{dottedzzznameb}\verb+>+}  \\
\end{tabular}

\subsection{global\_stmt}
\label{globalzzzstmtb}
\begin{tabular}{lcl}
{\bf \verb+<+global\_stmt\verb+>+} & ::=  & global {\bf \verb+<+ident~\ref{identb}\verb+>+}  \\
 & \verb+|+  & {\bf \verb+<+global\_stmt~\ref{globalzzzstmtb}\verb+>+}  \verb|,| {\bf \verb+<+ident~\ref{identb}\verb+>+}  \\
\end{tabular} \\

\urlh{http://www.python.org/doc/current/ref/global.html}{Global Statement\\ (\path{\1})}

The global statement means that the listed identifiers are to be interpreted as globals.
Names listed in a global statement must not be used in the same code block 
textually preceding that global statement. Names listed in a global statement 
must not be defined as formal parameters or in a for loop control target, 
class definition, function definition, or import statement.  

Programmer's note: the global is a directive to the parser. 
It applies only to code parsed at the same time as the global statement. 
In particular, a global statement contained in an exec statement does not 
affect the code block containing the exec statement, and code contained in 
an exec statement is unaffected by global statements in the code containing 
the exec statement. The same applies to the eval(), execfile() and compile() 
functions.

For example, in:
\begin{verbatim}
exec "global x"
x = 2
\end{verbatim}
the generated code will define a new \verb|x|, not modify an existing one.

\subsection{exec\_stmt (NOT YET IMPLEMENTED)}
\label{execzzzstmtb}
\begin{tabular}{lcl}
{\bf \verb+<+exec\_stmt\verb+>+} & ::=  & exec {\bf \verb+<+expr~\ref{exprb}\verb+>+}  \\
 & \verb+|+  & exec {\bf \verb+<+expr~\ref{exprb}\verb+>+}  in {\bf \verb+<+test~\ref{testb}\verb+>+}  \\
 & \verb+|+  & exec {\bf \verb+<+expr~\ref{exprb}\verb+>+}  in {\bf \verb+<+test~\ref{testb}\verb+>+}  \verb|,| {\bf \verb+<+test~\ref{testb}\verb+>+}  \\
\end{tabular} \\

http://www.python.org/doc/current/ref/exec.html

\subsection{assert\_stmt}
\label{assertzzzstmtb}
\begin{tabular}{lcl}
{\bf \verb+<+assert\_stmt\verb+>+} & ::=  & assert {\bf \verb+<+test~\ref{testb}\verb+>+}  \\
 & \verb+|+  & assert {\bf \verb+<+test~\ref{testb}\verb+>+}  \verb|,| {\bf \verb+<+test~\ref{testb}\verb+>+}  \\
\end{tabular} \\

\path{http://www.python.org/doc/current/ref/assert.html}

The simple form, ``\verb|assert expression|'', is equivalent to 
\begin{verbatim}
if __debug__:
   if not expression: raise AssertionError
\end{verbatim}

 The extended form, ``\verb|assert expression1, expression2|'', is equivalent to 
\begin{verbatim}
if __debug__:
   if not expression1: raise AssertionError, expression2
\end{verbatim}

\subsection{compound\_stmt}
\label{compoundzzzstmtb}
\newlength{\tw}
\settowidth{\tw}{compound\_stmt  ::=  }
\addtolength{\tw}{2\arraycolsep}
\newlength{\len}
\setlength{\len}{\textwidth}
\addtolength{\len}{-1\tw}
\begin{tabular}{lcp{\len}}
{\bf \verb+<+compound\_stmt\verb+>+} & ::=  & {\bf \verb+<+if\_stmt~\ref{ifzzzstmtb}\verb+>+}  \\
 & \verb+|+  & {\bf \verb+<+while\_stmt~\ref{whilezzzstmtb}\verb+>+}  \\
 & \verb+|+  & {\bf \verb+<+for\_stmt~\ref{forzzzstmtb}\verb+>+}  \\
 & \verb+|+  & {\bf \verb+<+try\_stmt~\ref{tryzzzstmtb}\verb+>+}  \\
 & \verb+|+  & {\bf \verb+<+funcdef~\ref{funcdefb}\verb+>+}  \\
 & \verb+|+  & {\bf \verb+<+classdef~\ref{classdefb}\verb+>+}  \\
\end{tabular} \\

Compound statements can span multiple lines, so they may introduce a new indentation level.

\subsection{if\_stmt}
\label{ifzzzstmtb}
\newlength{\tw}
\settowidth{\tw}{if\_stmt  ::=  }
\addtolength{\tw}{2\arraycolsep}
\newlength{\len}
\setlength{\len}{\textwidth}
\addtolength{\len}{-1\tw}
\begin{tabular}{lcp{\len}}
{\bf \verb+<+if\_stmt\verb+>+} & ::=  & if {\bf \verb+<+test~\ref{testb}\verb+>+}  \verb|:| {\bf \verb+<+suite~\ref{suiteb}\verb+>+}  {\bf \verb+<+elif\_list~\ref{elifzzzlistb}\verb+>+}  \\
 & \verb+|+  & if {\bf \verb+<+test~\ref{testb}\verb+>+}  \verb|:| {\bf \verb+<+suite~\ref{suiteb}\verb+>+}  {\bf \verb+<+elif\_list~\ref{elifzzzlistb}\verb+>+}  else \verb|:| {\bf \verb+<+suite~\ref{suiteb}\verb+>+}  \\
\end{tabular}

\subsection{elif\_list}
\label{elifzzzlistb}
\begin{tabular}{lcl}
{\bf \verb+<+elif\_list\verb+>+} & ::=  & $\varepsilon$ \\
 & \verb+|+  & {\bf \verb+<+elif\_list~\ref{elifzzzlistb}\verb+>+}  elif {\bf \verb+<+test~\ref{testb}\verb+>+}  \verb|:| {\bf \verb+<+suite~\ref{suiteb}\verb+>+}  \\
\end{tabular}

%the suite subsection was moved
\subsection{suite}
\label{suiteb}
\begin{tabular}{lcl}
{\bf \verb+<+suite\verb+>+} & ::=  & {\bf \verb+<+simple\_stmt~\ref{simplezzzstmtb}\verb+>+}  \\
 & \verb+|+  & NEWLINE INDENT {\bf \verb+<+stmt\_list\_plus~\ref{stmtzzzlistzzzplusb}\verb+>+}  DEDENT \\
\end{tabular} \\

The INDENT token indicates a new indentation level.  Similarly, the DEDENT token indicates a return to the previous indentation level.

\subsection{stmt\_list\_plus}
\label{stmtzzzlistzzzplusb}
\begin{tabular}{lcl}
{\bf \verb+<+stmt\_list\_plus\verb+>+} & ::=  & {\bf \verb+<+stmt~\ref{stmtb}\verb+>+}  \\
 & \verb+|+  & {\bf \verb+<+stmt\_list\_plus~\ref{stmtzzzlistzzzplusb}\verb+>+}  {\bf \verb+<+stmt~\ref{stmtb}\verb+>+}  \\
\end{tabular}

\subsection{while\_stmt}
\label{whilezzzstmtb}
\newlength{\tw}
\settowidth{\tw}{while\_stmt  ::=  }
\addtolength{\tw}{2\arraycolsep}
\newlength{\len}
\setlength{\len}{\textwidth}
\addtolength{\len}{-1\tw}
\begin{tabular}{lcp{\len}}
{\bf \verb+<+while\_stmt\verb+>+} & ::=  & while {\bf \verb+<+test~\ref{testb}\verb+>+}  \verb|:| {\bf \verb+<+suite~\ref{suiteb}\verb+>+}  \\
 & \verb+|+  & while {\bf \verb+<+test~\ref{testb}\verb+>+}  \verb|:| {\bf \verb+<+suite~\ref{suiteb}\verb+>+}  else \verb|:| {\bf \verb+<+suite~\ref{suiteb}\verb+>+}  \\
\end{tabular}

\subsection{for\_stmt}
\label{forzzzstmtb}
\newlength{\tw}
\settowidth{\tw}{for\_stmt  ::=  }
\addtolength{\tw}{2\arraycolsep}
\newlength{\len}
\setlength{\len}{\textwidth}
\addtolength{\len}{-1\tw}
\begin{tabular}{lcp{\len}}
{\bf \verb+<+for\_stmt\verb+>+} & ::=  & for {\bf \verb+<+target\_tuple\_or\_expr~\ref{targetzzztuplezzzorzzzexprb}\verb+>+}  in {\bf \verb+<+tuple\_or\_test~\ref{tuplezzzorzzztestb}\verb+>+}  \verb|:| {\bf \verb+<+suite~\ref{suiteb}\verb+>+}  \\
 & \verb+|+  & for {\bf \verb+<+target\_tuple\_or\_expr~\ref{targetzzztuplezzzorzzzexprb}\verb+>+}  in {\bf \verb+<+tuple\_or\_test~\ref{tuplezzzorzzztestb}\verb+>+}  \verb|:| {\bf \verb+<+suite~\ref{suiteb}\verb+>+}  else \verb|:| {\bf \verb+<+suite~\ref{suiteb}\verb+>+}  \\
\end{tabular}

\subsection{try\_stmt}
\label{tryzzzstmtb}
\newlength{\tw}
\settowidth{\tw}{try\_stmt  ::=  }
\addtolength{\tw}{2\arraycolsep}
\newlength{\len}
\setlength{\len}{\textwidth}
\addtolength{\len}{-1\tw}
\begin{tabular}{lcp{\len}}
{\bf \verb+<+try\_stmt\verb+>+} & ::=  & try \verb|:| {\bf \verb+<+suite~\ref{suiteb}\verb+>+}  {\bf \verb+<+except\_clause\_list\_plus~\ref{exceptzzzclausezzzlistzzzplusb}\verb+>+}  \\
 & \verb+|+  & try \verb|:| {\bf \verb+<+suite~\ref{suiteb}\verb+>+}  {\bf \verb+<+except\_clause\_list\_plus~\ref{exceptzzzclausezzzlistzzzplusb}\verb+>+}  else \verb|:| {\bf \verb+<+suite~\ref{suiteb}\verb+>+}  \\
 & \verb+|+  & try \verb|:| {\bf \verb+<+suite~\ref{suiteb}\verb+>+}  finally \verb|:| {\bf \verb+<+suite~\ref{suiteb}\verb+>+}  \\
\end{tabular}

\subsection{except\_clause\_list\_plus}
\label{exceptzzzclausezzzlistzzzplusb}
\newlength{\tw}
\settowidth{\tw}{except\_clause\_list\_plus  ::=  }
\addtolength{\tw}{2\arraycolsep}
\newlength{\len}
\setlength{\len}{\textwidth}
\addtolength{\len}{-1\tw}
%Yes, except that I didn't know \let existed (I use \newlength).  Also
%I don't see why you need the \verb and the "+", and you should also
%not forget that that are spaces added between the columns (see
%\arraycolsep).
\begin{tabular}{lcp{\len}}
{\bf \verb+<+except\_clause\_list\_plus\verb+>+} & ::=  & {\bf \verb+<+except\_clause~\ref{exceptzzzclauseb}\verb+>+}  \verb|:| {\bf \verb+<+suite~\ref{suiteb}\verb+>+}  \\
 & \verb+|+  & {\bf \verb+<+except\_clause\_list\_plus~\ref{exceptzzzclausezzzlistzzzplusb}\verb+>+}  {\bf \verb+<+except\_clause~\ref{exceptzzzclauseb}\verb+>+}  \verb|:| {\bf \verb+<+suite~\ref{suiteb}\verb+>+}  \\
\end{tabular}

\subsection{except\_clause}
\label{exceptzzzclauseb}
\begin{tabular}{lcl}
{\bf \verb+<+except\_clause\verb+>+} & ::=  & except \\
 & \verb+|+  & except {\bf \verb+<+test~\ref{testb}\verb+>+}  \\
 & \verb+|+  & except {\bf \verb+<+test~\ref{testb}\verb+>+}  \verb|,| {\bf \verb+<+test~\ref{testb}\verb+>+}  \\
\end{tabular}

\subsection{funcdef}
\label{funcdefb}
\urlh{http://www.python.org/doc/current/ref/function.html}{Function Definition\\ (\path{\1})} \\

\begin{tabular}{lcl}
{\bf \verb+<+funcdef\verb+>+} & ::=  & def {\bf \verb+<+ident~\ref{identb}\verb+>+}  {\bf \verb+<+parameters~\ref{parametersb}\verb+>+}  \verb|:| {\bf \verb+<+suite~\ref{suiteb}\verb+>+}  \\
\end{tabular}

\subsection{parameters}
\label{parametersb}
\begin{tabular}{lcl}
{\bf \verb+<+parameters\verb+>+} & ::=  & \verb|(| \verb|)| \\
 & \verb+|+  & \verb|(| {\bf \verb+<+varargslist~\ref{varargslistb}\verb+>+}  \verb|)| \\
\end{tabular}

\subsection{varargslist}
\label{varargslistb}
\begin{tabular}{lcl}
{\bf \verb+<+varargslist\verb+>+} & ::=  & \verb|**| {\bf \verb+<+ident~\ref{identb}\verb+>+}  \\
 & \verb+|+  & \verb|*| {\bf \verb+<+ident~\ref{identb}\verb+>+}  \\
 & \verb+|+  & \verb|*| {\bf \verb+<+ident~\ref{identb}\verb+>+}  \verb|,| \verb|**| {\bf \verb+<+ident~\ref{identb}\verb+>+}  \\
 & \verb+|+  & {\bf \verb+<+fpdef~\ref{fpdefb}\verb+>+}  \verb|,| \\
 & \verb+|+  & {\bf \verb+<+fpdef~\ref{fpdefb}\verb+>+}  \verb|=| {\bf \verb+<+test~\ref{testb}\verb+>+}  \verb|,| \\
 & \verb+|+  & {\bf \verb+<+fpdef~\ref{fpdefb}\verb+>+}  \\
 & \verb+|+  & {\bf \verb+<+fpdef~\ref{fpdefb}\verb+>+}  \verb|=| {\bf \verb+<+test~\ref{testb}\verb+>+}  \\
 & \verb+|+  & {\bf \verb+<+fpdef~\ref{fpdefb}\verb+>+}  \verb|,| {\bf \verb+<+varargslist~\ref{varargslistb}\verb+>+}  \\
 & \verb+|+  & {\bf \verb+<+fpdef~\ref{fpdefb}\verb+>+}  \verb|=| {\bf \verb+<+test~\ref{testb}\verb+>+}  \verb|,| {\bf \verb+<+varargslist~\ref{varargslistb}\verb+>+}  \\
\end{tabular} \\

If the form ``*identifier'' is present, it is initialized to a tuple receiving 
any excess positional parameters, defaulting to the empty tuple. 
If the form ``**identifier'' is present, it is initialized to a new dictionary 
receiving any excess keyword arguments, defaulting to a new empty dictionary.

If a parameter has a default value, all following parameters must also have a 
default value -- this is a syntactic restriction that is not expressed by the grammar.   
Default parameter values are evaluated when the function definition is executed. 
This means that the expression is evaluated once, when the function is defined, 
and that that same ``pre-computed'' value is used for each call. This is especially 
important to understand when a default parameter is a mutable object, such as a list 
or a dictionary: if the function modifies the object (e.g. by appending an item to 
a list), the default value is in effect modified. This is generally not what was 
intended. A way around this is to use None as the default, and explicitly test for 
it in the body of the function, e.g.:
\begin{verbatim}
def whats_on_the_telly(penguin=None):
    if penguin is None:
         penguin = []    
         penguin.append("property of the zoo")    
         return penguin
\end{verbatim}

\subsection{fpdef}
\label{fpdefb}
\begin{tabular}{lcl}
{\bf \verb+<+fpdef\verb+>+} & ::=  & {\bf \verb+<+ident~\ref{identb}\verb+>+}  \\
 & \verb+|+  & ( {\bf \verb+<+fplist~\ref{fplistb}\verb+>+}  ) \\
\end{tabular} \\

A function parameter is either an identifier or a tuple that will be unpacked.  For example, in:
\begin{verbatim}
def f(x, (y, z)):
    pass

f(1,(2,3))
\end{verbatim}
when f is called, \verb|x| is bound to \verb|1|, \verb|y| is bound to \verb|2|, and
\verb|z| is bound to \verb|3|. \\

\subsection{fplist}
\label{fplistb}
\begin{tabular}{lcl}
{\bf \verb+<+fplist\verb+>+} & ::=  & {\bf \verb+<+fpdef~\ref{fpdefb}\verb+>+}  \\
 & \verb+|+  & {\bf \verb+<+fpdef~\ref{fpdefb}\verb+>+}  \verb|,| \\
 & \verb+|+  & {\bf \verb+<+fpdef~\ref{fpdefb}\verb+>+}  \verb|,| {\bf \verb+<+fplist~\ref{fplistb}\verb+>+}  \\
\end{tabular}

\subsection{classdef}
\label{classdefb}
\begin{tabular}{lcl}
{\bf \verb+<+classdef\verb+>+} & ::=  & class {\bf \verb+<+ident~\ref{identb}\verb+>+}  \verb|:| {\bf \verb+<+suite~\ref{suiteb}\verb+>+}  \\
 & \verb+|+  & class {\bf \verb+<+ident~\ref{identb}\verb+>+}  \verb|(| {\bf \verb+<+test~\ref{testb}\verb+>+}  \verb|)| \verb|:| {\bf \verb+<+suite~\ref{suiteb}\verb+>+}  \\
 & \verb+|+  & class {\bf \verb+<+ident~\ref{identb}\verb+>+}  \verb|(| {\bf \verb+<+testlist~\ref{testlistb}\verb+>+}  \verb|)| \verb|:| {\bf \verb+<+suite~\ref{suiteb}\verb+>+}  \\
\end{tabular} \\

\urlh{http://www.python.org/doc/current/ref/class.html}{Class Definitions\\ (\path{\1})}

In the CPython interpreter, ``old-style'' classes are defined by
\verb+class classname(superclasses)+, and ``new-style'' classes are defined by 
\verb+class classname+.  Such distinction is not made here; all classes
are ``new-style'' classes.

\section{Expressions}
Expressions produce a value that can be used in an assignment or a boolean check.

\subsection{tuple\_or\_test}
\label{tuplezzzorzzztestb}
\begin{tabular}{lcl}
{\bf \verb+<+tuple\_or\_test\verb+>+} & ::=  & {\bf \verb+<+tuple~\ref{tupleb}\verb+>+}  \\
 & \verb+|+  & {\bf \verb+<+test~\ref{testb}\verb+>+}  \\
\end{tabular}

\subsection{tuple}
\label{tupleb}
\begin{tabular}{lcl}
{\bf \verb+<+tuple\verb+>+} & ::=  & {\bf \verb+<+testlist~\ref{testlistb}\verb+>+}  \\
\end{tabular} \\

A tuple has the same syntactic rules as a testlist (\ref{testlistb}), but the name ``tuple'' is more descriptive sometimes.

\subsection{testlist}
\label{testlistb}
\begin{tabular}{lcl}
{\bf \verb+<+testlist\verb+>+} & ::=  & {\bf \verb+<+test~\ref{testb}\verb+>+}  \verb|,| \\
 & \verb+|+  & {\bf \verb+<+test~\ref{testb}\verb+>+}  \verb|,| {\bf \verb+<+test~\ref{testb}\verb+>+}  \\
 & \verb+|+  & {\bf \verb+<+test~\ref{testb}\verb+>+}  \verb|,| {\bf \verb+<+testlist~\ref{testlistb}\verb+>+}  \\
\end{tabular}

\subsection{test}
\label{testb}
\urlh{http://www.python.org/doc/current/ref/Booleans.html}{Python Ref: Booleans\\ (\path{\1})}

\begin{tabular}{lcl}
{\bf \verb+<+test\verb+>+} & ::=  & {\bf \verb+<+or\_test~\ref{orzzztestb}\verb+>+}  \\
 & \verb+|+  & {\bf \verb+<+lambdef~\ref{lambdefb}\verb+>+}  \\
\end{tabular}

\subsection{lambdef}
\label{lambdefb}
\begin{tabular}{lcl}
{\bf \verb+<+lambdef\verb+>+} & ::=  & lambda {\bf \verb+<+varargslist~\ref{varargslistb}\verb+>+}  \verb|:| {\bf \verb+<+test~\ref{testb}\verb+>+}  \\
 & \verb+|+  & lambda \verb|:| {\bf \verb+<+test~\ref{testb}\verb+>+}  \\
\end{tabular}

\subsection{or\_test}
\label{orzzztestb}
\begin{tabular}{lcl}
{\bf \verb+<+or\_test\verb+>+} & ::=  & {\bf \verb+<+and\_test~\ref{andzzztestb}\verb+>+}  \\
 & \verb+|+  & {\bf \verb+<+or\_test~\ref{orzzztestb}\verb+>+}  or {\bf \verb+<+and\_test~\ref{andzzztestb}\verb+>+}  \\
\end{tabular}

\subsection{and\_test}
\label{andzzztestb}
\begin{tabular}{lcl}
{\bf \verb+<+and\_test\verb+>+} & ::=  & {\bf \verb+<+not\_test~\ref{notzzztestb}\verb+>+}  \\
 & \verb+|+  & {\bf \verb+<+and\_test~\ref{andzzztestb}\verb+>+}  and {\bf \verb+<+not\_test~\ref{notzzztestb}\verb+>+}  \\
\end{tabular}

\subsection{not\_test}
\label{notzzztestb}
\begin{tabular}{lcl}
{\bf \verb+<+not\_test\verb+>+} & ::=  & not {\bf \verb+<+not\_test~\ref{notzzztestb}\verb+>+}  \\
 & \verb+|+  & {\bf \verb+<+comparison~\ref{comparisonb}\verb+>+}  \\
\end{tabular}

\subsection{comparison}
\label{comparisonb}
\begin{tabular}{lcl}
{\bf \verb+<+comparison\verb+>+} & ::=  & {\bf \verb+<+expr~\ref{exprb}\verb+>+}  \\
 & \verb+|+  & {\bf \verb+<+comparison~\ref{comparisonb}\verb+>+}  {\bf \verb+<+comp\_op~\ref{compzzzopb}\verb+>+}  {\bf \verb+<+expr~\ref{exprb}\verb+>+}  \\
\end{tabular}

\subsection{comp\_op}
\label{compzzzopb}
\begin{tabular}{lcl}
{\bf \verb+<+comp\_op\verb+>+} & ::=  & \verb|<| \\
 & \verb+|+  & \verb|>| \\
 & \verb+|+  & \verb|==| \\
 & \verb+|+  & \verb|>=| \\
 & \verb+|+  & \verb|<=| \\
 & \verb+|+  & \verb|<>| \\
 & \verb+|+  & \verb|!=| \\
 & \verb+|+  & in \\
 & \verb+|+  & not in \\
 & \verb+|+  & is \\
 & \verb+|+  & is not \\
\end{tabular}

\subsection{expr}
\label{exprb}
\begin{tabular}{lcl}
{\bf \verb+<+expr\verb+>+} & ::=  & {\bf \verb+<+xor\_expr~\ref{xorzzzexprb}\verb+>+}  \\
 & \verb+|+  & {\bf \verb+<+expr~\ref{exprb}\verb+>+}  \verb+|+ {\bf \verb+<+xor\_expr~\ref{xorzzzexprb}\verb+>+}  \\
\end{tabular}

\subsection{xor\_expr}
\label{xorzzzexprb}
\begin{tabular}{lcl}
{\bf \verb+<+xor\_expr\verb+>+} & ::=  & {\bf \verb+<+and\_expr~\ref{andzzzexprb}\verb+>+}  \\
 & \verb+|+  & {\bf \verb+<+xor\_expr~\ref{xorzzzexprb}\verb+>+}  \verb|^| {\bf \verb+<+and\_expr~\ref{andzzzexprb}\verb+>+}  \\
\end{tabular}

\subsection{and\_expr}
\label{andzzzexprb}
\begin{tabular}{lcl}
{\bf \verb+<+and\_expr\verb+>+} & ::=  & {\bf \verb+<+shift\_expr~\ref{shiftzzzexprb}\verb+>+}  \\
 & \verb+|+  & {\bf \verb+<+and\_expr~\ref{andzzzexprb}\verb+>+}  \verb|&| {\bf \verb+<+shift\_expr~\ref{shiftzzzexprb}\verb+>+}  \\
\end{tabular}

\subsection{shift\_expr}
\label{shiftzzzexprb}
\begin{tabular}{lcl}
{\bf \verb+<+shift\_expr\verb+>+} & ::=  & {\bf \verb+<+arith\_expr~\ref{arithzzzexprb}\verb+>+}  \\
 & \verb+|+  & {\bf \verb+<+shift\_expr~\ref{shiftzzzexprb}\verb+>+}  \verb|<<| {\bf \verb+<+arith\_expr~\ref{arithzzzexprb}\verb+>+}  \\
 & \verb+|+  & {\bf \verb+<+shift\_expr~\ref{shiftzzzexprb}\verb+>+}  \verb|>>| {\bf \verb+<+arith\_expr~\ref{arithzzzexprb}\verb+>+}  \\
\end{tabular}

\subsection{arith\_expr}
\label{arithzzzexprb}
\begin{tabular}{lcl}
{\bf \verb+<+arith\_expr\verb+>+} & ::=  & {\bf \verb+<+term~\ref{termb}\verb+>+}  \\
 & \verb+|+  & {\bf \verb+<+arith\_expr~\ref{arithzzzexprb}\verb+>+}  \verb|+| {\bf \verb+<+term~\ref{termb}\verb+>+}  \\
 & \verb+|+  & {\bf \verb+<+arith\_expr~\ref{arithzzzexprb}\verb+>+}  \verb|-| {\bf \verb+<+term~\ref{termb}\verb+>+}  \\
\end{tabular}

\subsection{term}
\label{termb}
\begin{tabular}{lcl}
{\bf \verb+<+term\verb+>+} & ::=  & {\bf \verb+<+factor~\ref{factorb}\verb+>+}  \\
 & \verb+|+  & {\bf \verb+<+term~\ref{termb}\verb+>+}  \verb|*| {\bf \verb+<+factor~\ref{factorb}\verb+>+}  \\
 & \verb+|+  & {\bf \verb+<+term~\ref{termb}\verb+>+}  \verb|/| {\bf \verb+<+factor~\ref{factorb}\verb+>+}  \\
 & \verb+|+  & {\bf \verb+<+term~\ref{termb}\verb+>+}  \verb|%| {\bf \verb+<+factor~\ref{factorb}\verb+>+}  \\
 & \verb+|+  & {\bf \verb+<+term~\ref{termb}\verb+>+}  \verb|//| {\bf \verb+<+factor~\ref{factorb}\verb+>+}  \\
\end{tabular}

\subsection{factor}
\label{factorb}
\begin{tabular}{lcl}
{\bf \verb+<+factor\verb+>+} & ::=  & \verb|+| {\bf \verb+<+factor~\ref{factorb}\verb+>+}  \\
 & \verb+|+  & \verb|-| {\bf \verb+<+factor~\ref{factorb}\verb+>+}  \\
 & \verb+|+  & \verb|~| {\bf \verb+<+factor~\ref{factorb}\verb+>+}  \\
 & \verb+|+  & {\bf \verb+<+power~\ref{powerb}\verb+>+}  \\
\end{tabular}

\subsection{power}
\label{powerb}
\begin{tabular}{lcl}
{\bf \verb+<+power\verb+>+} & ::=  & {\bf \verb+<+atom~\ref{atomb}\verb+>+}  {\bf \verb+<+trailer\_list~\ref{trailerzzzlistb}\verb+>+}  \\
 & \verb+|+  & {\bf \verb+<+atom~\ref{atomb}\verb+>+}  {\bf \verb+<+trailer\_list~\ref{trailerzzzlistb}\verb+>+}  \verb|**| {\bf \verb+<+factor~\ref{factorb}\verb+>+}  \\
\end{tabular} \\

A trailer is an indexing operator.  (e.g., the \verb|[7]| in \verb|x[7]|).

\subsection{trailer\_list}
\label{trailerzzzlistb}
\begin{tabular}{lcl}
{\bf \verb+<+trailer\_list\verb+>+} & ::=  & $\varepsilon$ \\
 & \verb+|+  & {\bf \verb+<+trailer~\ref{trailerb}\verb+>+}  {\bf \verb+<+trailer\_list~\ref{trailerzzzlistb}\verb+>+}  \\
\end{tabular}

\subsection{trailer}
\label{trailerb}
\begin{tabular}{lcl}
{\bf \verb+<+trailer\verb+>+} & ::=  & ( ) \\
 & \verb+|+  & ( {\bf \verb+<+arglist~\ref{arglistb}\verb+>+}  ) \\
 & \verb+|+  & [ {\bf \verb+<+subscriptlist~\ref{subscriptlistb}\verb+>+}  ] \\
 & \verb+|+  & [ {\bf \verb+<+subscript~\ref{subscriptb}\verb+>+}  ] \\
 & \verb+|+  & \verb|.| {\bf \verb+<+ident~\ref{identb}\verb+>+}  \\
\end{tabular}

%arglist has been moved
\subsection{arglist}
\label{arglistb}
\begin{tabular}{lcl}
{\bf \verb+<+arglist\verb+>+} & ::=  & \verb|**| {\bf \verb+<+test~\ref{testb}\verb+>+}  \\
 & \verb+|+  & \verb|*| {\bf \verb+<+test~\ref{testb}\verb+>+}  \\
 & \verb+|+  & \verb|*| {\bf \verb+<+test~\ref{testb}\verb+>+}  \verb|,| \verb|**| {\bf \verb+<+test~\ref{testb}\verb+>+}  \\
 & \verb+|+  & {\bf \verb+<+argument~\ref{argumentb}\verb+>+}  \\
 & \verb+|+  & {\bf \verb+<+argument~\ref{argumentb}\verb+>+}  \verb|,| \\
 & \verb+|+  & {\bf \verb+<+argument~\ref{argumentb}\verb+>+}  \verb|,| {\bf \verb+<+arglist~\ref{arglistb}\verb+>+}  \\
\end{tabular}

\subsection{atom}
\label{atomb}
\begin{tabular}{lcl}
{\bf \verb+<+atom\verb+>+} & ::=  & ( {\bf \verb+<+tuple\_or\_test~\ref{tuplezzzorzzztestb}\verb+>+}  ) \\
 & \verb+|+  & [ {\bf \verb+<+listmaker~\ref{listmakerb}\verb+>+}  ] \\
 & \verb+|+  & \{ {\bf \verb+<+dictmaker~\ref{dictmakerb}\verb+>+}  \} \\
 & \verb+|+  & ( ) \\
 & \verb+|+  & [ ] \\
 & \verb+|+  & \{ \} \\
 & \verb+|+  & ` {\bf \verb+<+tuple\_or\_test~\ref{tuplezzzorzzztestb}\verb+>+}  ` \\
 & \verb+|+  & {\bf \verb+<+ident~\ref{identb}\verb+>+}  \\
 & \verb+|+  & NUMBER \\
 & \verb+|+  & {\bf \verb+<+string\_list\_plus~\ref{stringzzzlistzzzplusb}\verb+>+}  \\
\end{tabular} \\

\begin{itemize}
\item ( ) is the empty tuple
\item \verb|[ ]| is the empty list
\item \{ \} is the empty dictionary
\item \verb|`|\ldots\verb|`| is a shortcut for \verb|repr(|\ldots\verb|)|
\end{itemize}

\subsection{string\_list\_plus}
\label{stringzzzlistzzzplusb}
\begin{tabular}{lcl}
{\bf \verb+<+string\_list\_plus\verb+>+} & ::=  & STRING \\
 & \verb+|+  & STRING {\bf \verb+<+string\_list\_plus~\ref{stringzzzlistzzzplusb}\verb+>+}  \\
\end{tabular}

\subsection{listmaker}
\label{listmakerb}
\begin{tabular}{lcl}
{\bf \verb+<+listmaker\verb+>+} & ::=  & {\bf \verb+<+test~\ref{testb}\verb+>+}  {\bf \verb+<+list\_for~\ref{listzzzforb}\verb+>+}  \\
 & \verb+|+  & {\bf \verb+<+testlist~\ref{testlistb}\verb+>+}  \\
 & \verb+|+  & {\bf \verb+<+test~\ref{testb}\verb+>+}  \\
\end{tabular}

\subsection{list\_iter}
\label{listzzziterb}
\begin{tabular}{lcl}
{\bf \verb+<+list\_iter\verb+>+} & ::=  & {\bf \verb+<+list\_for~\ref{listzzzforb}\verb+>+}  \\
 & \verb+|+  & {\bf \verb+<+list\_if~\ref{listzzzifb}\verb+>+}  \\
\end{tabular}

\subsection{list\_for}
\label{listzzzforb}
\newlength{\tw}
\settowidth{\tw}{list\_for  ::=  }
\addtolength{\tw}{2\arraycolsep}
\newlength{\len}
\setlength{\len}{\textwidth}
\addtolength{\len}{-1\tw}
\begin{tabular}{lcp{\len}}
{\bf \verb+<+list\_for\verb+>+} & ::=  & for {\bf \verb+<+target\_tuple\_or\_expr~\ref{targetzzztuplezzzorzzzexprb}\verb+>+}  in {\bf \verb+<+testlist\_safe~\ref{testlistzzzsafeb}\verb+>+}  \\
 & \verb+|+  & for {\bf \verb+<+target\_tuple\_or\_expr~\ref{targetzzztuplezzzorzzzexprb}\verb+>+}  in {\bf \verb+<+testlist\_safe~\ref{testlistzzzsafeb}\verb+>+}  {\bf \verb+<+list\_iter~\ref{listzzziterb}\verb+>+}  \\
\end{tabular} \\

\urlh{http://www.python.org/doc/current/ref/lists.html}{List Comprehensions\\ (\path{\1})}

\subsection{list\_if}
\label{listzzzifb}
\begin{tabular}{lcl}
{\bf \verb+<+list\_if\verb+>+} & ::=  & if {\bf \verb+<+test~\ref{testb}\verb+>+}  \\
 & \verb+|+  & if {\bf \verb+<+test~\ref{testb}\verb+>+}  {\bf \verb+<+list\_iter~\ref{listzzziterb}\verb+>+}  \\
\end{tabular}

\subsection{subscriptlist}
\label{subscriptlistb}
\begin{tabular}{lcl}
{\bf \verb+<+subscriptlist\verb+>+} & ::=  & {\bf \verb+<+subscript~\ref{subscriptb}\verb+>+}  \verb|,| \\
 & \verb+|+  & {\bf \verb+<+subscript~\ref{subscriptb}\verb+>+}  \verb|,| {\bf \verb+<+subscript~\ref{subscriptb}\verb+>+}  \\
 & \verb+|+  & {\bf \verb+<+subscript~\ref{subscriptb}\verb+>+}  \verb|,| {\bf \verb+<+subscriptlist~\ref{subscriptlistb}\verb+>+}  \\
\end{tabular}

%the last production had to be changed from \verb|.| \verb|.| \verb|.| to \verb|...|
\subsection{subscript}
\label{subscriptb}
\begin{tabular}{lcl}
{\bf \verb+<+subscript\verb+>+} & ::=  & \verb|:| \\
 & \verb+|+  & \verb|:| {\bf \verb+<+test~\ref{testb}\verb+>+}  \\
 & \verb+|+  & {\bf \verb+<+test~\ref{testb}\verb+>+}  \verb|:| \\
 & \verb+|+  & {\bf \verb+<+test~\ref{testb}\verb+>+}  \verb|:| {\bf \verb+<+test~\ref{testb}\verb+>+}  \\
 & \verb+|+  & \verb|:| {\bf \verb+<+sliceop~\ref{sliceopb}\verb+>+}  \\
 & \verb+|+  & \verb|:| {\bf \verb+<+test~\ref{testb}\verb+>+}  {\bf \verb+<+sliceop~\ref{sliceopb}\verb+>+}  \\
 & \verb+|+  & {\bf \verb+<+test~\ref{testb}\verb+>+}  \verb|:| {\bf \verb+<+sliceop~\ref{sliceopb}\verb+>+}  \\
 & \verb+|+  & {\bf \verb+<+test~\ref{testb}\verb+>+}  \verb|:| {\bf \verb+<+test~\ref{testb}\verb+>+}  {\bf \verb+<+sliceop~\ref{sliceopb}\verb+>+}  \\
 & \verb+|+  & {\bf \verb+<+test~\ref{testb}\verb+>+}  \\
 & \verb+|+  & \verb|...| \\
\end{tabular}

\subsection{sliceop}
\label{sliceopb}
\begin{tabular}{lcl}
{\bf \verb+<+sliceop\verb+>+} & ::=  & \verb|:| \\
 & \verb+|+  & \verb|:| {\bf \verb+<+test~\ref{testb}\verb+>+}  \\
\end{tabular}

\subsection{exprlist}
\label{exprlistb}
\begin{tabular}{lcl}
{\bf \verb+<+exprlist\verb+>+} & ::=  & {\bf \verb+<+expr~\ref{exprb}\verb+>+}  \verb|,| \\
 & \verb+|+  & {\bf \verb+<+expr~\ref{exprb}\verb+>+}  \verb|,| {\bf \verb+<+expr~\ref{exprb}\verb+>+}  \\
 & \verb+|+  & {\bf \verb+<+expr~\ref{exprb}\verb+>+}  \verb|,| {\bf \verb+<+exprlist~\ref{exprlistb}\verb+>+}  \\
\end{tabular}

\subsection{target\_tuple\_or\_expr}
\label{targetzzztuplezzzorzzzexprb}
\begin{tabular}{lcl}
{\bf \verb+<+target\_tuple\_or\_expr\verb+>+} & ::=  & {\bf \verb+<+target\_tuple~\ref{targetzzztupleb}\verb+>+}  \\
 & \verb+|+  & {\bf \verb+<+expr~\ref{exprb}\verb+>+}  \\
\end{tabular}

\subsection{target\_tuple}
\label{targetzzztupleb}
\begin{tabular}{lcl}
{\bf \verb+<+target\_tuple\verb+>+} & ::=  & {\bf \verb+<+exprlist~\ref{exprlistb}\verb+>+}  \\
\end{tabular}

\subsection{testlist\_safe}
\label{testlistzzzsafeb}
\begin{tabular}{lcl}
{\bf \verb+<+testlist\_safe\verb+>+} & ::=  & {\bf \verb+<+test~\ref{testb}\verb+>+}  \\
 & \verb+|+  & {\bf \verb+<+test~\ref{testb}\verb+>+}  \verb|,| {\bf \verb+<+testlist~\ref{testlistb}\verb+>+}  \\
\end{tabular} \\

Used in list comprehensions (\ref{listzzzforb}).

\subsection{dictmaker}
\label{dictmakerb}
\begin{tabular}{lcl}
{\bf \verb+<+dictmaker\verb+>+} & ::=  & {\bf \verb+<+test~\ref{testb}\verb+>+}  \verb|:| {\bf \verb+<+test~\ref{testb}\verb+>+}  \\
 & \verb+|+  & {\bf \verb+<+test~\ref{testb}\verb+>+}  \verb|:| {\bf \verb+<+test~\ref{testb}\verb+>+}  \verb|,| \\
 & \verb+|+  & {\bf \verb+<+test~\ref{testb}\verb+>+}  \verb|:| {\bf \verb+<+test~\ref{testb}\verb+>+}  \verb|,| {\bf \verb+<+dictmaker~\ref{dictmakerb}\verb+>+}  \\
\end{tabular}

\subsection{argument}
\label{argumentb}
\begin{tabular}{lcl}
{\bf \verb+<+argument\verb+>+} & ::=  & {\bf \verb+<+test~\ref{testb}\verb+>+}  \\
 & \verb+|+  & {\bf \verb+<+ident~\ref{identb}\verb+>+}  \verb|=| {\bf \verb+<+test~\ref{testb}\verb+>+}  \\
\end{tabular}

\end{document}
