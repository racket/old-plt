\input tex2page
\input css.t2p
\texonly
\def\evalsto{\leavevmode\hbox{$\Rightarrow$}}
\endtexonly

\htmlonly
\def\evalsto{=>}
\endhtmlonly

\let\q\scmp

\scmkeyword{function}

\title{scm2cl}

\ignorenextinputtimestamp

\centerline{\urlh{scmxlate.tar.gz}{\htmlonly
Download \p{scmxlate}\endhtmlonly, version \input version }}
\centerline{(\p{scm2cl} is part of the \urlh{scmxlate.html}{scmxlate} distribution)}

\centerline{\urlh{../index.html}{Dorai Sitaram}}

\medskip

\p{scm2cl} is a Common Lisp
program that converts Scheme
code to Common Lisp.  Its goal is to remove the tedium of
transcribing a large Scheme file into CL, not to avoid human
intervention altogether.  You may have to do some
hand-tuning --- \p{scm2cl} will assist you here too.

\p{scm2cl} understands R5RS Scheme, including the
high-level macros (\q{define-syntax}, \q{let-syntax},
and \q{letrec-syntax}), plus some common Scheme
extensions, viz, \q{andmap}, \q{append!},
\q{call-with-input-string},
\q{call-with-output-string}, \q{delete-file}, \q{exit},
\q{file-exists?}, \q{file-or-directory-modify-seconds},
\q{fluid-let}, \q{flush-output}, \q{getenv}, \q{get-output-string},
\q{open-input-string}, \q{open-output-string},
\q{ormap}, \q{read-line}, \q{reverse!},
\q{string-index}, \q{string-reverse-index}, \q{system}.

\section{Usage}

With some Common Lisp dialects, you can call \p{scm2cl.cl}
on the operating-system command-line, with the Scheme file
to be converted as argument:

\verb{
scm2cl.cl scmfile.scm
}

\n This produces the CL file in \p{scmfile.cl}.

If this is not possible, you may load \p{scm2cl.cl} into
your Common Lisp and invoke the function \q{scheme-to-cl}.
Eg,

\scm{
(load "scm2cl.cl")
(scheme-to-cl "scmfile.scm")
}

\section{Human Intervention}

\subsection{\q{funcall} and \q{function}}

\p{scm2cl} is good at inserting \q{funcall} and \q{function}
(aka \q{#'}) where needed, but not perfect.  It will,
however, draw your attention to code locations where a
\q{funcall} {\it may\/} be needed.  \p{scm2cl} will not
miss any of these problem spots, so you don't have to hunt
for \q{funcall} locations on your own.

No similar signaling is offered for potential
\q{#'} locations.
(The reason is that this would lead to too many signals for
\q{#'}, the vast majority of them being red herrings.
Detecting \q{funcall} is less noisy, but still not perfect
--- which is why \p{scm2cl} suggests passively instead of
inserting actively.)

\subsection{Some additional definitions}

The CL file produced by \p{scm2cl} may use some of the
procedures defined in the file
\p{scheme-procs.cl}.  (\p{scheme-procs.cl}  contains some Scheme-like
procedure definitions for CL.)  At the end of the
conversion, \p{scm2cl} will give you the exact list of
Scheme procedures from
\p{scheme-procs.cl} actually used.  You will need to
incorporate these procedure definitions into the
converted code.  How you do it is up to your
convenience.

\subsection{Caveats}

Only ``escape'' continuations are allowed for
\q{call-with-current-continuation}.

``Named'' \q{let}s whose names begin with the substring
\q{"loop"} are converted into iterative CL loops.
\p{scm2cl} assumes that such named \q{let}s were intended to
write iterative (``tail-recursive'') loops, and translates
accordingly.  This will cause error if the \q{let} wasn't so
intended.

Named \q{let}s with names that do not begin with \q{loop}
are converted using CL \q{labels}.

%\section{References}
%
%\nocite{clhs,cltl}
%
%\bibliographystyle{plain}
%\bibliography{scm2cl}

\bye
