% tex2html.tex
% Dorai Sitaram, Apr 1997

\message{version 3p}

% TeX files using these macros
% can be converted by the program
% tex2html into HTML

\let\texonly\relax
\let\endtexonly\relax

\texonly

\ifx\slatexignorecurrentfile\UNDEFINED\relax\fi

\def\defcsactive#1{\defnumactive{`#1}}

\def\defnumactive#1#2{\catcode#1\active
  \begingroup\lccode`\~#1%
    \lowercase{\endgroup\def~{#2}}}

% gobblegobblegobble

\def\gobblegroup{\bgroup
  \def\do##1{\catcode`##1=9 }\dospecials
  \catcode`\{1 \catcode`\}2 \catcode`\^^M=9
  \gobblegroupI}

\def\gobblegroupI#1{\egroup}

\def\gobbleencl{\bgroup
  \def\do##1{\catcode`##1=12 }\dospecials
  \catcode`\{1 \catcode`\}2 \catcode`\^^M=9
  \futurelet\gobbleenclnext\gobbleenclI}

\def\gobbleenclI{\ifx\gobbleenclnext\bgroup
    \let\gobbleenclnext\gobblegroupI
  \else\let\gobbleenclnext\gobbleenclII\fi
  \gobbleenclnext}

\def\gobbleenclII#1{%
  \def\gobbleenclIII##1#1{\egroup}%
  \gobbleenclIII}

% \verb
% Usage: \verb{...lines...} or \verb|...lines...|
% In the former case, | can be used as escape char within
% the verbatim text

\let\verbhook\relax

\def\verbfont{\tt}
%\hyphenchar\tentt-1

\def\verbsetup{\frenchspacing
  \def\do##1{\catcode`##1=12 }\dospecials
  \catcode`\|=12 % needed?
  \verbfont}

% The current font is cmtt iff fontdimen3 = 0 _and_
% fontdimen7 != 0

\def\checkifusingcmtt{\let\usingcmtt n%
  \ifdim\the\fontdimen3\the\font=0.0pt
    \ifdim\the\fontdimen7\the\font=0.0pt
    \else\let\usingcmtt y\fi\fi}

% In a nonmonospaced font, - followed by a letter
% is a regular hyphen.  Followed by anything else, it is a
% typewriter hyphen.

\def\variablelengthhyphen{\futurelet\variablelengthhyphenI
  \variablelengthhyphenII}

\def\variablelengthhyphenII{\ifcat\noexpand\variablelengthhyphenI
 a-\else{\tt\char`\-}\fi}

\def\verbavoidligs{% avoid ligatures
  \defcsactive\`{\relax\lq}%
  \defcsactive\ {\leavevmode\ }%
  \defcsactive\^^I{\leavevmode\ \ \ \ \ \ \ \ }%
  \defcsactive\^^M{\leavevmode\endgraf}%
  \checkifusingcmtt
  \ifx\usingcmtt n%
  \defcsactive\<{\relax\char`\<}%
  \defcsactive\>{\relax\char`\>}%
  \defcsactive\-{\variablelengthhyphen}%
  \fi}

\def\verbinsertskip{%
  \let\firstpar y%
  \defcsactive\^^M{\ifx\firstpar y%
    \let\firstpar n%
    \verbdisplayskip
    \aftergroup\verbdisplayskip
    \else\leavevmode\fi\endgraf}%
  \verbhook}

\def\verb{\begingroup
  \verbsetup\verbI}

\newcount\verbbracebalancecount

\def\verblbrace{\char`\{}
\def\verbrbrace{\char`\}}

\def\verbescapechar#1{%
  \def\escapifyverbescapechar{\catcode`#1=0 }}

\verbescapechar\|

{\catcode`\[1 \catcode`\]2
\catcode`\{12 \catcode`\}12
\gdef\verbI#1[\verbavoidligs
  \verbinsertskip\verbhook
  \if#1{\escapifyverbescapechar
    \def\{[\char`\{]%
    \def\}[\char`\}]%
    \def\|[\char`\|]%
    \verbbracebalancecount0
    \defcsactive\{[\advance\verbbracebalancecount by 1
      \verblbrace]%
    \defcsactive\}[\ifnum\verbbracebalancecount=0
      \let\verbrbracenext\endgroup\else
      \advance\verbbracebalancecount by -1
      \let\verbrbracenext\verbrbrace\fi
      \verbrbracenext]\else
  \defcsactive#1[\endgroup]\fi
  \verbII
]]

\def\verbII{\futurelet\verbIInext\verbIII}

{\catcode`\^^M\active%
\gdef\verbIII{\ifx\verbIInext^^M\else%
  \defcsactive\^^M{\leavevmode\ }\fi}}

\let\verbdisplayskip\medbreak

% \verbinput FILENAME
% displays contents of file FILENAME verbatim.

\def\verbinput#1 {{\verbsetup\verbavoidligs\verbhook
  \input #1 }}

\def\verbfilename#1 {\relax}
\let\verbwrite\gobbleencl

% \path is like \verb except that its argument
% can break across lines at `.' and `/'.

\def\path{\begingroup\verbsetup
  \pathfont
  \defcsactive\.{\discretionary{\char`\.}{}{\char`\.}}%
  \defcsactive\/{\discretionary{\char`\/}{}{\char`\/}}%
  \verbI}

\let\pathfont\relax

% \url{URL} becomes
% <a href="URL">URL</a> in HTML, and
% URL in DVI.

% A-VERY-VERY-LONG-URL in a .bib file
% could be split by BibTeX
% across a linebreak, with % before the newline.
% To accommodate this, %-followed-by-newline will
% be ignored in the URL argument of \url and related
% macros.

\def\url{\bgroup\urlsetup\let\dummy=}

\def\urlsetup{\verbsetup\urlfont\verbavoidligs
  \catcode`\{1 \catcode`\}2
  \defcsactive\%{\urlpacifybibtex}%
  \defcsactive\ {\relax}%
  \defcsactive\^^M{\relax}%
  \defcsactive\.{\discretionary{\char`\.}{}{\char`\.}}%
  \defcsactive\/{\discretionary{\char`\/}{}{\char`\/}}%
  \defcsactive\`{\relax\lq}}

\let\urlfont\relax

\def\urlpacifybibtex{\futurelet\urlpacifybibtexnext\urlpacifybibtexI}

\def\urlpacifybibtexI{\ifx\urlpacifybibtexnext^^M%
  \else\%\fi}

% \mailto{ADDRESS} becomes
% <a href="mailto:ADDRESS">ADDRESS</a> in HTML, and
% ADDRESS in DVI.

\let\mailto\url

% \urlh{URL}{TEXT} becomes
% <a href="URL">TEXT</a> in HTML, and
% TEXT in DVI.

% If TEXT contains \\, the part after \\ appears in
% the DVI only.  If, further, this part contains \1,
% the latter is replaced by a fixed-width representation
% of URL.

\def\urlh{\bgroup\urlsetup
  \afterassignment\urlhI
  \gdef\urlII}

\def\urlhI{\egroup
  \bgroup
    \let\\\relax
    \def\1{{\urlsetup\urlII}}%
    \let\dummy=}

% \urlhd{URL}{HTML-TEXT}{DVI-TEXT} becomes
% <a href="URL">HTML-TEXT</a> in HTML, and
% DVI-TEXT in DVI

\def\urlhd{\bgroup
  \def\do##1{\catcode`##1=12 }\dospecials
  \catcode`\{1 \catcode`\}2
  \urlhdI}

\def\urlhdI#1#2{\egroup}

%

\let\ignorenextinputtimestamp\relax

%

\let\htmlonly\iffalse
\let\endhtmlonly\fi

\def\rawhtml{\errmessage{Can't occur except inside
  \string\htmlonly}}
\def\endrawhtml{\errmessage{Can't occur except inside
  \string\htmlonly}}

\let\htmlheadonly\iffalse
\let\endhtmlheadonly\fi

\let\htmlstylesheet\gobblegroup

% color (deprecated)

\let\rgb\gobblegroup
\let\color\gobblegroup

% Scheme

\let\scm\verb
\let\scminput\verbatiminput

\def\scmfilename#1 {\relax}
\let\scmdribble\scm
\let\scmwrite\gobbleencl

\let\scmkeyword\gobblegroup
\let\setkeyword\gobblegroup % SLaTeX compat

\ifx\slatexversion\UNDEFINED
\def\schemedisplay{\begingroup
  \verbsetup\verbavoidligs
  \verbinsertskip
  \schemedisplayI}%
\fi

{\catcode`\|0 |catcode`|\12
  |long|gdef|schemedisplayI#1\endschemedisplay{%
    #1|endgroup}}

% GIFs

\let\gifdef\def

\def\gifpreamble{\let\magnificationoutsidegifpreamble\magnification
  \def\magnification{\count255=}}

\def\endgifpreamble{\let\magnification\magnificationoutsidegifpreamble}

\let\htmlgif\relax
\let\endhtmlgif\relax

% Cheap count registers: doesn't use up TeX's limited
% number of real count registers.

% A cheap count register is simply a macro that expands to the
% contents of the count register.  Thus \def\kount{0} defines a
% count register \kount that currently contains 0.

% \advancecheapcount\kount num increments \kount by n.
% \globaladvancecheapcount increments the global \kount.
% If \kount is not defined, the \[global]advancecheapcount
% macros define it to be 0 before proceeding with the
% incrementation.

\def\newcheapcount#1{\edef#1{0}}

\def\advancecheapcounthelper#1#2#3{%
  \ifx#2\UNDEFINED
    #1\edef#2{0}\fi
  \edef\setcountCCLV{\count255=#2 }%
  \setcountCCLV
  \advance\count255 by #3
  #1\edef#2{\the\count255 }}

\def\advancecheapcount{\advancecheapcounthelper\relax}
\def\globaladvancecheapcount{\advancecheapcounthelper\global}

% title

\let\title\gobblegroup

\def\subject#1{\centerline{\bf#1}\medskip}

% plain's \beginsection splits pages too easily

%\def\beginsection#1\par{\sectionwithnumber{1}{}{#1}}

\def\beginsection{\vskip-\lastskip
  \bigbreak\noindent
  \bgroup\bf
    \let\par\sectionafterskip}

\def\beginsectionstar*{\beginsection}

% plain's \{left,center,right}line can't handle catcode change
% within their argument

\def\leftline{\line\bgroup\bgroup
  \aftergroup\leftlinefinish
  \let\dummy=}

\def\leftlinefinish{\hss\egroup}

\def\centerline{\line\bgroup\bgroup
  \aftergroup\leftlinefinish
  \hss\let\dummy=}

\def\rightline{\line\bgroup\hss\let\dummy=}

%

\let\strike\fiverm % can be much better!

%

\let\htmlpagebreak\relax

\let\htmlpagelabel\gobblegroup

\def\htmlpageref{\errmessage{Can't occur except inside
  \string\htmlonly}}

% Miscellaneous stuff

\def\hr{$$\hbox{---}$$}
\def\hr{\medbreak\centerline{---}\medbreak}
%\def\hr{\par\centerline{$*$}\par}
%\def\hr{\smallskip\line{\leaders\hbox{~.~}\hfill}\smallskip}

%Commonplace math that doesn't require GIF.  (Avoiding $
%here because $ triggers GIF generation.)

\def\mathg{$\bgroup\aftergroup\closemathg\let\dummy=}
\def\closemathg{$}

\def\mathdg{$$\bgroup\aftergroup\closemathdg\let\dummy=}
\def\closemathdg{$$}

\def\frac#1/#2{{#1\over#2}}

%

% Backward compatible stuff

\let\p\verb
\let\verbatim\verb
\let\verbatimfile\verbinput
\let\setverbatimescapechar\verbescapechar
\let\scmp\scm
\let\scmverbatim\scm
\let\scmverbatimfile\scminput
\let\scmfile\scmdribble
\let\scmfileonly\scmwrite
\let\href\urlhd

\endtexonly

\ifx\newenvironment\UNDEFINED\else
% we're in LaTeX and so won't load rest of file
\endinput\fi

\texonly

\input btxmac

% Sections

\def\tracksectionchangeatlevel#1{%
  \expandafter\let\expandafter\thiscount\csname
    sectionnumber#1\endcsname
  \ifx\thiscount\relax
    \expandafter\edef\csname sectionnumber#1\endcsname{0}%
  \fi
  \expandafter\advancecheapcount
    \csname sectionnumber#1\endcsname 1%
  \ifx\doingappendix0%
    \edef\recentlabel{\csname sectionnumber1\endcsname}%
  \else
    %\count255=\expandafter\csname sectionnumber1\endcsname
    \edef\recentlabel{\char\csname sectionnumber1\endcsname}%
  \fi
  \count255=0
  \loop
    \advance\count255 by 1
    \ifnum\count255=1
    \else\edef\recentlabel{\recentlabel.\csname
      sectionnumber\the\count255\endcsname}\fi
  \ifnum\count255<#1%
  \repeat
  \loop
    \advance\count255 by 1
    \expandafter\let\expandafter\nextcount\csname
      sectionnumber\the\count255\endcsname
    \ifx\nextcount\relax
      \let\continue0%
    \else
      \expandafter\edef\csname
        sectionnumber\the\count255\endcsname{0}%
      \let\continue1\fi
  \ifx\continue1%
  \repeat}

% Vanilla section-header look -- change this macro for new look

\def\sectionstar#1*#2{\vskip-\lastskip
  % #1=depth #2=heading-text
  \tocactivate
  {\let\folio0%
   \edef\temp{\write\tocout{\string\tocentry{#1}{}{#2}{\folio}}}%
   \temp}%
  \goodbreak
  \vskip1.5\bigskipamount
  \noindent
  \hbox{\bf\vtop{\hsize=.7\hsize
    \pretolerance 10000
    \noindent\raggedright#2}}%
  \bgroup\let\par\sectionafterskip}

\def\sectionwithnumber#1#2#3{\vskip-\lastskip
  % #1=depth #2=dotted-number #3=heading-text
  \tocactivate
  {\let\folio0%
   \edef\temp{\write\tocout{\string\tocentry{#1}{#2}{#3}{\folio}}}%
   \temp}
  \goodbreak
  \vskip1.5\bigskipamount
  \noindent
  \hbox{\bf#2\vtop{\hsize=.7\hsize
    \pretolerance 10000
    \noindent\raggedright#3}}%
  \bgroup\let\par\sectionafterskip}

% \edef\temp{\write\tocout{\string\hskip#1\space em\string\relax\space #2%
%    \string\vtop{\string\hsize=.7\string\hsize
%    \string\noindent\string\raggedright\space #3}\string\par}}\temp

\def\sectionafterskip{\egroup\nobreak\medskip\noindent}

\def\sectiond#1{\count255=#1%
  \ifx\usingchapters1\advance\count255 by 1 \fi
  \edef\sectiondlvl{\the\count255 }%
  \futurelet\sectionnextchar\sectiondispatch}

\def\sectiondispatch{\ifx\sectionnextchar*%
  \def\sectioncontinue{\sectionstar{\sectiondlvl}}\else
  \tracksectionchangeatlevel{\sectiondlvl}
  \def\sectioncontinue{\sectionwithnumber{\sectiondlvl}%
    {\recentlabel\enspace}}\fi
  \sectioncontinue}

\def\section{\sectiond1}
\def\subsection{\sectiond2}
\def\subsubsection{\sectiond3}
\def\paragraph{\sectiond4}
\def\subparagraph{\sectiond5}

\let\usingchapters0

\def\chapter{\global\let\usingchapters1%
\futurelet\chapternextchar\chapterdispatch}

\def\chapterdispatch{\ifx\chapternextchar*%
  \let\chaptercontinue\chapterstar\else
  \tracksectionchangeatlevel{1}%
  \def\chaptercontinue{\chapterhelp{\recentlabel}}\fi
  \chaptercontinue}

\def\chapterstar*#1{%
  % #1=heading-text
  \tocactivate
  {\let\folio0%
   \edef\temp{\write\tocout{\string\tocentry{1}{}{#1}{\folio}}}%
   \temp}%
  \vfill\eject
  \null\vskip3em
  \noindent
  \hbox{\bf\vtop{\hsize=.7\hsize
    \pretolerance 10000
    \noindent\raggedright#1}}%
  \bgroup\let\par\chapterafterskip}

\def\chapterhelp#1#2{%
  % #1=number #2=heading-text
  \tocactivate
  {\let\folio0%
   \edef\temp{\write\tocout{\string\tocentry{1}{#1\enspace}{#2}{\folio}}}%
   \temp}%
  \vfill\eject
  \null\vskip3em
  \noindent
  \ifx\doingappendix0%
    \hbox{\bf Chapter #1}\else
    \hbox{\bf Appendix #1}\fi
  \vskip 1em
  \noindent
  \hbox{\bf\vtop{\hsize=.7\hsize
    \pretolerance 10000
    \noindent\raggedright#2}}%
  \bgroup\let\par\chapterafterskip}

\def\chapterafterskip{\egroup\nobreak\vskip3em \noindent}

\let\doingappendix=0
\def\appendix{\let\doingappendix=1%
  \count255=`\A%
  \advance\count255 by -1
  \expandafter\edef\csname
    sectionnumber1\endcsname{\the\count255 }}

% toc

\let\tocactive0

\def\tocoutensure{\ifx\tocout\UNDEFINED
  \csname newwrite\endcsname\tocout\fi}

\def\tocactivate{\ifx\tocactive0%
  \tocoutensure
  \tocsave
  \openout\tocout \jobname.toc
  \global\let\tocactive1\fi}

\def\tocspecials{\def\do##1{\catcode`##1=12 }\dospecials}

\def\tocsave{\openin0=\jobname.toc
  \ifeof0 \closein0 \else
    \openout\tocout Z-T-\jobname.tex
    \let\tocsaved 0%
    \loop
      \ifeof0 \closeout\tocout
        \let\tocsaved1%
      \else{\tocspecials
         \read0 to \tocsaveline
         \edef\temp{\write\tocout{\tocsaveline}}\temp}%
      \fi
    \ifx\tocsaved0%
    \repeat
  \fi
  \closein0 }

\def\tocentry#1#2#3#4{%
  %#1=depth #2=secnum #3=sectitle #4=secpage
  \ifnum#1=1\medbreak\begingroup\bf
  \else\begingroup\fi
  \noindent\hskip #1 em
  #2%
  \vtop{\hsize=.7\hsize
    \raggedright
    \noindent {#3},
    #4\strut}\endgroup\par}

\def\tableofcontents{%
  \ifx\tocactive0%
    \openin0 \jobname.toc
    \ifeof0 \closein0 \else
      \closein0 \input \jobname.toc
    \fi
    \tocoutensure
    \openout\tocout \jobname.toc
    \global\let\tocactive1%
  \else
    \input Z-T-\jobname.tex
  \fi}

% Cross-references

% \openxrefout loads all the TAG-VALUE associations in
% \jobname.xrf and then opens \jobname.xrf as an
% output channel that \tag can use

\def\openxrefout{\openin0=\jobname.xrf
  \ifeof0 \closein0 \else
  {\catcode`\\0 \input \jobname.xrf }\fi
  \csname newwrite\endcsname\xrefout
  \openout\xrefout=\jobname.xrf }

% \tag{TAG}{VALUE} associates TAG with VALUE.
% Hereafter, \ref{TAG} will output VALUE.
% \tag stores its associations in \xrefout.
% \tag calls \openxrefout if \jobname.xrf hasn't
% already been opened

\def\tag#1#2{\ifx\xrefout\UNDEFINED\openxrefout\fi
  {\let\folio0%
    \edef\temp{%
      \write\xrefout{\string\expandafter\string\gdef
        \string\csname\space XREF#1\string\endcsname
        {#2}\string\relax}}%
    \temp}}

% \ref{TAG} outputs VALUE, assuming \tag put such
% an association into \xrefout.  \ref calls
% \openxrefout if \jobname.xrf hasn't already
% been opened

\def\ref#1{\ifx\xrefout\UNDEFINED\openxrefout\fi
  \expandafter\ifx\csname XREF#1\endcsname\relax
  %\message or \write16 ?
  \message{\the\inputlineno: Unresolved label `#1'.}?\else
  \csname XREF#1\endcsname\fi}

% \label, as in LaTeX

\let\recentlabel\relax

% The sectioning commands
% define \recentlabel so a subsequent call to \label will pick up the
% right label.

\def\label#1{\tag{#1}{\recentlabel}%
  \tag{PAGE#1}{\folio}}

% \pageref, as in LaTeX

\def\pageref#1{\ref{PAGE#1}}

% Numbered footnotes

\newcheapcount\footnotenumber

\ifx\plainfootnote\UNDEFINED
  \let\plainfootnote\footnote
\fi

\def\numfootnote{\globaladvancecheapcount\footnotenumber 1%
  \bgroup\csname footnotehook\endcsname
    \plainfootnote{$^{\footnotenumber}$}\bgroup
      \edef\recentlabel{\footnotenumber}%
      \aftergroup\egroup
      \let\dummy=}

%

\def\iffileexists#1#2#3{%
  \openin0 #1
  \ifeof0 \closein0
    #3%
  \else \closein0
    #2\fi}

% \ifx\bibitem\UNDEFINED
% \newcheapcount\bibitemnumber

% \def\bibitem{\par\globaladvancecheapcount\bibitemnumber 1%
% \edef\recentlabel{\bibitemnumber}%
% [\bibitemnumber]\label}
% \fi

%

% \def\begin#1{\begingroup
% \def\end##1{\csname end#1\endcsname\endgroup}%
% \def\envname{#1}%
% \def\envnameI{thebibliography}%
% \csname #1\endcsname
% \ifx\envname\envnameI\let\next\gobblegroup
% \else\let\next\relax\fi\next}

% \def\begin#1{\begingroup
% \let\end\endbegin
% \csname #1\endcsname}

% \def\endbegin#1{\csname end#1\endcsname\endgroup}

% Index generation
%
% Your TeX source contains \index{NAME} to
% signal that NAME should be included in the index.
% Check the makeindex documentation to see the various
% ways NAME can be specified, e.g., for subitems, for
% explicitly specifying the alphabetization for a name
% involving TeX control sequences, etc.
%
% The first run of TeX will create \jobname.idx.
% makeindex on \jobname[.idx] will create the sorted
% index \jobname.ind.
%
% Use \inputindex (without arguments) to include this
% sorted index, typically somewhere to the end of your
% document.  This will produce the items and subitems.
% It won't produce a section heading however -- you
% will have to typeset one yourself.
%
% Use \printindex instead of \inputindex if you want
% the section heading ``Index'' automatically generated.

\def\sanitizeidxletters{\def\do##1{\catcode`##1=11 }%
  \do\\\do\$\do\&\do\#\do\^\do\_\do\%\do\~%
  \do\@\do\"\do\!\do\|\do\-\do\ \do\'}

\def\index{%\unskip
  \ifx\indexout\UNDEFINED
    \csname newwrite\endcsname\indexout
    \openout\indexout \jobname.idx\fi
  \begingroup
    \sanitizeidxletters
    \indexI}

\def\indexI#1{\endgroup
  \write\indexout{\string\indexentry{#1}{\folio}}%
  \ignorespaces}

% The following index style indents subitems on a
% separate lines

\def\theindex{\begingroup
  \parskip0pt \parindent0pt
  \def\indexitem##1{\par\hangindent30pt \hangafter1
    \hskip ##1 }%
  \def\item{\indexitem{0em}}%
  \def\subitem{\indexitem{2em}}%
  \def\subsubitem{\indexitem{4em}}%
  \let\indexspace\medskip}

\def\endtheindex{\endgroup}

% \packindex declares that subitems be bundled into one
% semicolon-separated paragraph

\def\packindex{%
  \def\theindex{\begingroup
    \parskip0pt \parindent0pt
    \def\item{\par\hangindent20pt \hangafter1 }%
    \def\subitem{\unskip; }%
    \def\subsubitem{\unskip; }%
    \let\indexspace\medskip}}

\def\inputindex{%
  \openin0 \jobname.ind
  \ifeof0 \closein0
    \message{\jobname.ind missing.}%
  \else\closein0
    \begingroup
      \def\begin##1{\csname##1\endcsname}%
      \def\end##1{\csname end##1\endcsname}%
      \input\jobname.ind
    \endgroup\fi}

\def\printindex{\csname beginsection\endcsname Index\par
  \inputindex}

%

\def\italiccorrection{\futurelet\italiccorrectionI
  \italiccorrectionII}

\def\italiccorrectionII{%
  \if\noexpand\italiccorrectionI,\else
  \if\noexpand\italiccorrectionI.\else
  \/\fi\fi}

\def\em{\it\ifmmode\else\aftergroup\italiccorrection\fi}

%\def\emph{\bgroup\it
%  \ifmmode\else\aftergroup\italiccorrection\fi
%  \let\dummy=}

\def\itemize{\par\begingroup
  \advance\leftskip 1.5em
  \smallbreak
  \def\item{\smallbreak$\bullet$\enspace\ignorespaces}}

\def\enditemize{\smallbreak\smallbreak\endgroup\par}

\def\enumerate{\par\begingroup
  \newcheapcount\enumeratenumber
  \advance\leftskip 1.5em
  \smallbreak
  \def\item{\smallbreak
    \advancecheapcount\enumeratenumber1%
    {\bf \enumeratenumber.}\enspace\ignorespaces}}

\def\endenumerate{\smallbreak\smallbreak\endgroup\par}

\endtexonly

% end of file
