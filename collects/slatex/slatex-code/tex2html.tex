% tex2html.tex
% Dorai Sitaram, Apr 1997

% Plain TeX files loading and using these macros
% can be converted by the program
% ``tex2html'' into HTML

\let\texonly\relax
\let\endtexonly\relax

\texonly

\message{[ver. 2s]} % version

\let\gifdef\def

% verbatim environment
% Usage: \verbatim{...lines...} or \verbatim|...lines...|
% In the former case, | can be used as escape char within
% the verbatim text

\let\verbatimhook\relax

\def\verbatimfont{\tt}
%\hyphenchar\tentt-1

\def\defcsactive#1{\defnumactive{`#1}}

\def\defnumactive#1#2{\catcode#1\active
  \begingroup\lccode`\~#1%
    \lowercase{\endgroup\def~{#2}}}

\def\setupverbatim{%\frenchspacing
  \def\do##1{\catcode`##1=12 }\dospecials
  \catcode`\|=12
  \verbatimfont}

% The current font is cmtt iff fontdimen3 = 0 _and_
% fontdimen7 != 0

\def\checkifusingcmtt{\let\usingcmtt n%
  \ifdim\the\fontdimen3\the\font=0.0pt
    \ifdim\the\fontdimen7\the\font=0.0pt
    \else\let\usingcmtt y\fi\fi}

% In a nonmonospaced font, - followed by a letter
% is a hyphen.  Followed by anything else, it is a
% minus.

\def\variablelengthhyphen{\futurelet\variablelengthhyphenI
  \variablelengthhyphenII}

\def\variablelengthhyphenII{\ifcat\noexpand\variablelengthhyphenI
 a-\else{\tt\char`\-}\fi}

\def\avoidligs{% avoid ligatures
  \defcsactive\`{\relax\lq}%
  \defcsactive\ {\leavevmode\ }%
  \defcsactive\^^I{\leavevmode\ \ \ \ \ \ \ \ }%
  \checkifusingcmtt
  \ifx\usingcmtt n%
  \defcsactive\<{\relax\char`\<}%
  \defcsactive\>{\relax\char`\>}%
  \defcsactive\-{\variablelengthhyphen}%
  \fi
  \let\firstpar y%
  \defcsactive\^^M{\ifx\firstpar y%
    \let\firstpar n%
    \verbatimdisplayskip
    \aftergroup\verbatimdisplayskip
    \else\leavevmode\fi\endgraf}%
  \verbatimhook
}

\def\verbatim{\begingroup
  \setupverbatim\verbatimI}

\newcount\verbbalance

\def\verblbrace{\char`\{}
\def\verbrbrace{\char`\}}

\def\setverbatimescapechar#1{%
  \def\escapifyverbatimescapechar{\catcode`#1=0 }}

\setverbatimescapechar\|

{\catcode`\[1 \catcode`\]2
\catcode`\{12 \catcode`\}12
\gdef\verbatimI#1[\avoidligs
  \if#1{\escapifyverbatimescapechar
    %\catcode`\|0
    \def\{[\char`\{]%
    \def\}[\char`\}]%
    \def\|[\char`\|]%
    \verbbalance0
    \defcsactive\{[\advance\verbbalance by 1
      \verblbrace]%
    \defcsactive\}[\ifnum\verbbalance=0
      \let\verbrbracenext\endgroup\else\advance\verbbalance by -1
      \let\verbrbracenext\verbrbrace\fi
      \verbrbracenext]\else
  \defcsactive#1[\null\endgroup]\fi
  \verbatimII
]]

\def\verbatimII{\futurelet\verbatimIInext\verbatimIII}

{\catcode`\^^M\active%
\gdef\verbatimIII{\ifx\verbatimIInext^^M\else%
  \defcsactive\^^M{\leavevmode\ }\fi}}

\let\verbatimdisplayskip\medbreak

% use \p{code} or \p|code| for in-text code

\let\p\verbatim

% for URLs and such

\def\path{\begingroup\setupverbatim
\defcsactive\.{\discretionary{\char`\.}{}{\char`\.}}%
\defcsactive\/{\discretionary{\char`\/}{}{\char`\/}}%
\verbatimI}

% \verbatimfile filename
% inputs a file verbatim

\def\verbatimfile#1 {{\setupverbatim\avoidligs
\input #1 }}

%

\let\htmlonly\iffalse
\let\endhtmlonly\fi

\def\gobblegroup{\bgroup\def\do##1{\catcode`##1=12 }%
\dospecials\catcode`\{1 \catcode`\}2
\gobblegroupii}

\def\gobblegroupii#1{\egroup}

\let\rawhtml\gobblegroup

% color

\let\rgb\gobblegroup
\let\color\gobblegroup

% scheme

\let\scmp\verbatim
\let\scmverbatim\verbatim
\let\scmverbatimfile\verbatimfile

\let\scmkeyword\gobblegroup
\let\setkeyword\gobblegroup % SLaTeX compat

\ifx\slatexversion\UNDEFINED
\def\schemedisplay{\begingroup
  \setupverbatim\avoidligs
  \schemedisplayverbatim}
\fi

{\catcode`\|0 |catcode`|\12
|long|gdef|schemedisplayverbatim#1\endschemedisplay{%
#1|endgroup}}

% cross-references

% \openxrefout loads all the TAG-VALUE associations in
% \jobname.xrf and then opens \jobname.xrf as an
% output channel that \tag can use

\def\openxrefout{\openin0=\jobname.xrf
  \ifeof0 \closein0 \else
  {\catcode`\\0 \input \jobname.xrf }\fi
  \csname newwrite\endcsname\xrefout
  \openout\xrefout=\jobname.xrf }

% \tag{TAG}{VALUE} associates TAG with VALUE.
% Hereafter, \ref{TAG} will output VALUE
% \tag stores its associations in \xrefout.
% \tag calls \openxrefout if \jobname.xrf hasn't
% already been opened

\def\tag#1#2{\ifx\xrefout\UNDEFINED\openxrefout\fi
  {\let\folio0%
    \edef\temp{%
      \write\xrefout{\string\expandafter\string\gdef
        \string\csname\space XREF#1\string\endcsname
        {#2}\string\relax}}%
    \temp}}

% \ref{TAG} outputs VALUE, assuming \tag put such
% an association into \xrefout.  \ref calls
% \openxrefout if \jobname.xrf hasn't already
% been opened

\def\ref#1{\ifx\xrefout\UNDEFINED\openxrefout\fi
  \expandafter\ifx\csname XREF#1\endcsname\relax
  \write16{*** Unresolved xref #1 -- run TeX again.}?\else
  \csname XREF#1\endcsname\fi}

% \label, as in LaTeX

\let\recentlabel\relax

% The sectioning commands
% define \recentlabel so a subsequent call to \label will pick up the
% right label.

\def\label#1{\tag{#1}{\recentlabel}%
  \tag{PAGE#1}{\folio}}

% \pageref, as in LaTeX

\def\pageref#1{\ref{PAGE#1}}

% \url{URL} becomes
% <a href="URL">URL</a> in HTML, and
% \path{URL} in DVI

\def\url{\bgroup\it
\defcsactive\~{\~{}}%
\defcsactive\_{\_}%
\aftergroup\/
\let\dummy=}

\let\url\path

% \urlh{URL}{text} becomes
% <a href="URL">text</a> in HTML, and
% text in DVI

\def\urlh{\edef\restoresharptildecatcodes{%
  \noexpand\catcode`\noexpand\#=\the\catcode`\#%
  \noexpand\catcode`\noexpand\~=\the\catcode`\~}%
  \catcode`\#=12 \catcode`\~=12
\urlhI}

\def\urlhI#1{\restoresharptildecatcodes}

% \urlhd{URL}{HTML text}{DVI text} becomes
% <a href="URL">HTML text</a> in HTML, and
% {DVI text} in DVI

\def\urlhd{\edef\restoresharptildecatcodes{%
  \noexpand\catcode`\noexpand\#=\the\catcode`\#%
  \noexpand\catcode`\noexpand\~=\the\catcode`\~}%
  \catcode`\#=12 \catcode`\~=12
  \urlhdI}

\def\urlhdI#1#2{\restoresharptildecatcodes}

% gif

\def\gifpreamble{\let\magnificationoutsidegifpreamble\magnification
\def\magnification{\count255=}}
\def\endgifpreamble{\let\magnification\magnificationoutsidegifpreamble}

\let\htmlgif\relax
\let\endhtmlgif\relax

% Cheap count registers: doesn't use up TeX's limited
% number of real count registers.

% A cheap count register is simply a macro that expands to the
% contents of the count register.  Thus \def\kount{0} defines a
% count register \kount that currently contains 0.

% \advancecheapcount\kount num increments \kount by n.
% \globaladvancecheapcount increments the global \kount.
% If \kount is not defined, the \[global]advancecheapcount
% macros define it to be 0 before proceeding with the
% incrementation.

\def\newcheapcount#1{\edef#1{0}}

\def\advancecheapcounthelper#1#2#3{%
  \ifx#2\UNDEFINED
    #1\edef#2{0}\fi
  \edef\setcountCCLV{\count255=#2 }%
  \setcountCCLV
  \advance\count255 by #3
  #1\edef#2{\the\count255 }}

\def\advancecheapcount{\advancecheapcounthelper\relax}
\def\globaladvancecheapcount{\advancecheapcounthelper\global}

% title

\def\title#1{}

\def\subject#1{\centerline{\bf#1}\medskip}

% plain's \beginsection splits pages too easily

%\def\beginsection#1\par{\sectionwithnumber{1}{}{#1}}

\def\beginsection{\vskip-\lastskip
  \bigbreak\noindent
  \bgroup\bf
    \let\par\sectionafterskip}

\def\beginsectionstar*{\beginsection}

%

\let\strike\fiverm % can be much better!

%

\let\htmlpagebreak\relax

% Abbrevs

\let\n\noindent
\let\pp\verbatim
\let\q\scmp
\let\qq\scmverbatim

% Miscellaneous stuff

\def\hr{$$\hbox{---}$$}
\def\hr{\medbreak\centerline{---}\medbreak}
%\def\hr{\par\centerline{$*$}\par}
%\def\hr{\smallskip\line{\leaders\hbox{~.~}\hfill}\smallskip}

%Commonplace math that doesn't require GIF.  (Avoiding $
%here because $ triggers GIF generation.)

\def\mathg{$\bgroup\aftergroup\closemathg\let\dummy=}
\def\closemathg{$}

% bkwd compat stuf

\let\href\urlhd

% Don't load the following for LaTeX

\ifx\newenvironment\UNDEFINED\else\endinput\fi

%  sections

\def\tracksectionchangeatlevel#1{%
  \expandafter\let\expandafter\thiscount\csname
    sectionnumber#1\endcsname
  \ifx\thiscount\relax
    \expandafter\edef\csname sectionnumber#1\endcsname{0}%
  \fi
  \expandafter\advancecheapcount
    \csname sectionnumber#1\endcsname 1%
  \ifx\doingappendix0%
    \edef\recentlabel{\csname sectionnumber1\endcsname}%
  \else
    %\count255=\expandafter\csname sectionnumber1\endcsname
    \edef\recentlabel{\char\csname sectionnumber1\endcsname}%
  \fi
  \count255=0
  \loop
    \advance\count255 by 1
    \ifnum\count255=1
    \else\edef\recentlabel{\recentlabel.\csname
      sectionnumber\the\count255\endcsname}\fi
  \ifnum\count255<#1
  \repeat
  \loop
    \advance\count255 by 1
    \expandafter\let\expandafter\nextcount\csname
      sectionnumber\the\count255\endcsname
    \ifx\nextcount\relax
      \let\continue0
    \else
      \expandafter\edef\csname
        sectionnumber\the\count255\endcsname{0}%
      \let\continue1\fi
  \ifx\continue1
  \repeat}

% vanilla section-header look -- change this macro for new look

\let\generatetoc=0

\def\opentocoutifsafe{%
  \ifx\generatetoc0%
    \openin0=\jobname.toc
    \ifeof0 \closein0
      \csname newwrite\endcsname\tocout
      \openout\tocout=\jobname.toc
      \let\generatetoc=1%
    \fi
  \fi}

%\def\ensuretocout{\ifx\tocout\UNDEFINED
%  \csname newwrite\endcsname\tocout
%  \openout\tocout=\jobname.toc
%  \fi}

\def\sectionstar#1*#2{\vskip-\lastskip
  % #1=depth #2=heading-text
  \opentocoutifsafe
  \ifx\generatetoc1%
  \edef\temp{\write\tocout{\noindent\string\hskip#1\space em\string\relax%
    \string\vtop{\string\hsize=.7\string\hsize\string\raggedright
    \string\noindent\space #2\string\strut}\string\par}}\temp\fi
  \goodbreak
  \vksip1.5\bigskipamount
  \noindent
  \hbox{\bf\vtop{\hsize=.7\hsize
    \pretolerance 10000
    \noindent\raggedright#2}}%
  \bgroup\let\par\sectionafterskip}

\def\sectionwithnumber#1#2#3{\vskip-\lastskip
  % #1=depth #2=dotted-number #3=heading-text
  \opentocoutifsafe
  \ifx\generatetoc1%
  \edef\temp{\write\tocout{\noindent\string\hskip#1\space em\string\relax\space #2%
    \string\vtop{\string\hsize=.7\string\hsize\string\raggedright
    \string\noindent\space #3\string\strut}\string\par}}\temp\fi
  \goodbreak
  \vskip1.5\bigskipamount
  \noindent
  \hbox{\bf#2\vtop{\hsize=.7\hsize
    \pretolerance 10000
    \noindent\raggedright#3}}%
  \bgroup\let\par\sectionafterskip}

% \edef\temp{\write\tocout{\string\hskip#1\space em\string\relax\space #2%
%    \string\vtop{\string\hsize=.7\string\hsize
%    \string\noindent\string\raggedright\space #3}\string\par}}\temp

\def\sectionafterskip{\egroup\nobreak\medskip\noindent}

\def\sectiond#1{\def\sectiondlvl{#1}%
\futurelet\sectionnextchar\sectiondispatch}

\def\sectiondispatch{\ifx\sectionnextchar*%
  \def\sectioncontinue{\sectionstar{\sectiondlvl}}\else
  \tracksectionchangeatlevel{\sectiondlvl}
  \def\sectioncontinue{\sectionwithnumber{\sectiondlvl}%
    {\recentlabel.\enspace}}\fi
  \sectioncontinue}

%\def\sectiond#1{\tracksectionchangeatlevel{#1}%
%  \sectionwithnumber{#1}{\recentlabel.\enspace}}

\def\section{\sectiond1}
\def\subsection{\sectiond2}
\def\subsubsection{\sectiond3}
\def\subsubsubsection{\sectiond4}
\def\subsubsubsubsection{\sectiond5}

\def\chapter{\futurelet\chapternextchar\chapterdispatch}

\def\chapterdispatch{\ifx\chapternextchar*%
  \let\chaptercontinue\chapterstar\else
  \tracksectionchangeatlevel{1}%
  \def\chaptercontinue{\chapterhelp{\recentlabel}}\fi
  \chaptercontinue}

%\def\chapter{\tracksectionchangeatlevel{1}%
%  \chapterhelp{\recentlabel.\enspace}}

\def\chapterstar*#1{%
  % #1=heading-text
  \opentocoutifsafe
  \ifx\generatetoc1%
  \edef\temp{\write\tocout{\noindent\string\hskip1em\string\relax%
    \string\vtop{\string\hsize=.7\string\hsize\string\raggedright
    \string\noindent\space #1\string\strut}\string\par}}\temp\fi
  \vfill\eject
  \null\vskip3em
  \noindent
  \hbox{\bf\vtop{\hsize=.7\hsize
    \pretolerance 10000
    \noindent\raggedright#1}}%
  \bgroup\let\par\chapterafterskip}

\def\chapterhelp#1#2{%
  % #1=number #2=heading-text
  \opentocoutifsafe
  \ifx\generatetoc1%
  \edef\temp{\write\tocout{\noindent\string\hskip1em\string\relax\space #1.\enspace%
    \string\vtop{\string\hsize=.7\string\hsize\string\raggedright
    \string\noindent\space #2\string\strut}\string\par}}\temp\fi
  \vfill\eject
  \null\vskip3em
  \noindent
  \ifx\doingappendix0%
    \hbox{\bf Chapter #1}\else
    \hbox{\bf Appendix #1}\fi
  \vskip 1em
  \noindent
  \hbox{\bf\vtop{\hsize=.7\hsize
    \pretolerance 10000
    \noindent\raggedright#2}}%
  \bgroup\let\par\chapterafterskip}

\def\chapterafterskip{\egroup\nobreak\vskip3em \noindent}

\let\doingappendix=0
\def\appendix{\let\doingappendix=1%
  \count255=`\A%
  \advance\count255 by -1
  \expandafter\edef\csname
    sectionnumber1\endcsname{\the\count255 }}

\def\tableofcontents{%
  \openin0=\jobname.toc
  \ifeof0 \closein0 \else
  \message{Delete \jobname.toc to regenerate Table of
    Contents. }%
  \input \jobname.toc
  \fi
  %\csname newwrite\endcsname\tocout
  %\openout\tocout=\jobname.toc
}

% numbered footnotes

\newcheapcount\footnotenumber

\ifx\plainfootnote\UNDEFINED
\let\plainfootnote\footnote
\fi

\def\numfootnote{\globaladvancecheapcount\footnotenumber 1%
\bgroup\csname footnotehook\endcsname
\plainfootnote{$^{\footnotenumber}$}\bgroup
\edef\recentlabel{\footnotenumber}%
\aftergroup\egroup
\let\dummy=}

%

\newcheapcount\bibitemnumber

\def\bibitem{\par\globaladvancecheapcount\bibitemnumber 1%
\edef\recentlabel{\bibitemnumber}%
[\bibitemnumber]\label}

% Index generation.
%
% Your TeX source contains \index{name} to
% signal that name should be included in the index.
% Check the makeindex documentation to see the various
% ways `name' can be specified, e.g., for subitems, for
% explicitly specifying the alphabetization for a name
% involving TeX control sequences, &c.
%
% The first run of TeX will create \jobname.idx.
% makeindex on \jobname[.idx] will create the sorted
% index \jobname.ind.
%
% Use \inputindex (without arguments) to include this
% sorted index, typically somewhere to the end of your
% document.  This will produce the items and subitems.
% It won't produce a section heading however -- you
% will have to typeset one yourself.
%
% Use \printindex instead of \inputindex if you want
% the section heading ``Index'' automatically generated.

\def\sanitizeidxletters{\def\do##1{\catcode`##1=11 }%
\do\\\do\$\do\&\do\#\do\^\do\_\do\%\do\~%
\do\@\do\"\do\!\do\|\do\-\do\ \do\'}

\def\index{%\unskip
\ifx\indexout\UNDEFINED
\csname newwrite\endcsname\indexout
\openout\indexout \jobname.idx\fi
\begingroup
\sanitizeidxletters
\indexI}

\def\indexI#1{\endgroup
\write\indexout{\string\indexentry{#1}{\folio}}%
\ignorespaces
}

% The following index style indents subitems on a
% separate lines

\def\theindex{\begingroup
\parskip0pt \parindent0pt
\def\indexitem##1{\par\hangindent30pt \hangafter1
\hskip ##1 }%
\def\item{\indexitem{0em}}%
\def\subitem{\indexitem{2em}}%
\def\subsubitem{\indexitem{4em}}%
\let\indexspace\medskip}

\def\endtheindex{\endgroup}

% \packindex declares that subitems be bundled into one
% semicolon-separated paragraph

\def\packindex{%
%
\def\theindex{\begingroup
\parskip0pt \parindent0pt
\def\item{\par\hangindent20pt \hangafter1 }%
\def\subitem{\unskip; }%
\def\subsubitem{\unskip; }%
\let\indexspace\medskip}%
%
}

\def\inputindex{%
\openin0 \jobname.ind
\ifeof0 \closein0
\write16{*** Use makeindex to generate index file.}%
\else\closein0
\begingroup
\def\begin##1{\csname##1\endcsname}%
\def\end##1{\csname end##1\endcsname}%
\input\jobname.ind
\endgroup
\fi}

\def\printindex{\csname beginsection\endcsname Index\par
\inputindex}

%

\def\italiccorrection{\futurelet\italiccorrectionII
  \italiccorrectionIII}

\def\italiccorrectionIII{%
  \if\noexpand\italiccorrectionII,\else
  \if\noexpand\italiccorrectionII.\else
  \/\fi\fi}

\def\em{\it\ifmmode\else\aftergroup\italiccorrection\fi}

%\def\emph{\bgroup\it
%  \ifmmode\else\aftergroup\italiccorrection\fi
%  \let\dummy=}

\def\itemize{\par\begingroup
  \advance\leftskip 1.5em
  \smallbreak
  \def\item{\smallbreak$\bullet$\enspace\ignorespaces}}

\def\enditemize{\smallbreak\smallbreak\endgroup\par}

\def\enumerate{\par\begingroup
  \newcheapcount\enumeratenumber
  \advance\leftskip 1.5em
  \smallbreak
  \def\item{\smallbreak
    \advancecheapcount\enumeratenumber1%
    {\bf \enumeratenumber.}\enspace\ignorespaces}}

\def\endenumerate{\smallbreak\smallbreak\endgroup\par}

\endtexonly

%eof
