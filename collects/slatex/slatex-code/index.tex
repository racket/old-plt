% Webpage introduction to SLaTeX

\input tex2html

\gifpreamble
\magnification\magstep1
\endgifpreamble

\title{SLaTeX}

\subject{SLaTeX}

\centerline{\urlh{slatex.tar.gz}{[Download version \input version ]}}

\smallskip

\centerline{Dorai Sitaram}
\centerline{\urlh{mailto:ds26@gte.com}{ds26@gte.com}}

\bigskip

SLaTeX is a Scheme program that allows you to write program
code (or code fragments) ``as is'' in your LaTeX or plain
TeX source.  SLaTeX will typeset the code with appropriate
fonts for the various token categories --- e.g., boldface
for keywords and italics for variables ---, at the same time
retaining the proper indentations and vertical alignments in
TeX's non-monospace fonts.

For example, consider a LaTeX file \p{example.tex}
with the following contents:

\verbatim+
\documentclass{article}
\usepackage{slatex}
\begin{document}

In Scheme, the assignment expression \scheme|(set! answer 42)|
returns an unspecified value, rather than \scheme'42'.  We
could define a more Common-Lisp-like assignment operator as
follows:

\begin{schemedisplay}
(define-syntax setq
  (syntax-rules ()
    [(setq var val)
     (begin (set! var val)
            var)]))
\end{schemedisplay}

\end{document}
+

When run through SLaTeX, the resulting \p{example.dvi} file
looks as follows:

\hr

\htmlonly
\rawhtml{
In Scheme, the assignment expression
<code>(<strong>set!</strong> <em>answer</em> 42)</code>
returns an unspecified value, rather than <code>42</code>.
We could define a more Common-Lisp-like assignment operator
as follows:<p>

<pre>
(<strong>define-syntax setq</strong>
  (<strong>syntax-rules</strong> ()
    [(<strong>setq</strong> <em>var val</em>)
     (<strong>begin</strong> (<strong>set!</strong> <em>var val</em>)
            <em>var</em>)]))
</pre>
}\endhtmlonly

\hr

As the example shows, {\it in-text\/} code is introduced by
the control sequence \p{\scheme} and is flanked by either
identical characters or by matching braces.  Code meant for
{\it display\/} is presented between
\p{\begin{schemedisplay}} and
\p{\end{schemedisplay}}.  Note that you write the code
as you would when writing a program  --- no special
annotation is needed to get the typeset version.

SLaTeX recognizes the usual tokens of Scheme (or Common
Lisp, using an alternate suite that is provided in the
distribution).  It can also automatically learn about new
keywords as it comes across code containing macro
definitions (as for \p{setq} above).

Easy user commands are available for making SLaTeX recognize
a different default suite of tokens, or to employ different
font assignments.  In addition, the user can have SLaTeX
typeset certain identifiers as arbitrary TeX expressions
(e.g., \p{lambda} as the Greek letter $\lambda$).  More
comprehensive documentation of all that is possible with
SLaTeX is provided in the distribution.

Although SLaTeX is written in Scheme, a configuration
option is provided to make it run on Common Lisp.
SLaTeX has tested successfully on many different Scheme
and Common Lisp dialects, viz., Allegro Common Lisp,
Austin Kyoto Common Lisp, Bigloo, Chez Scheme, CLISP,
Elk, Gambit, Gnu Common Lisp, Guile, Ibuki Common Lisp,
Macintosh Common Lisp, MIT Scheme, MzScheme,
Scheme{\tt->}C, SCM, UMB Scheme, and VSCM.

\bye